\documentclass{article}

% Packages
\usepackage{fuzz}
\usepackage{geometry}
\usepackage{listings}

% Metadata
\title{SBM - Final Assignment}
\date{18th March 2025}
\author{Anonymous}

\begin{document}
\maketitle

\pagebreak 

% Question 1
\section*{Q1}

\subsection*{Part A}

The state model below does make multiple assumptions. As it was stated a timeline can be associated with at most one multiverse, I’ve assumed timelines don’t necessarily need to be linked to a multiverse when created and can be linked at a later point. I’ve also assumed that delete operations should clean up all associated states and not only be available when none exists.
\newline
\newline
The state model does introduce a timeline and character function through axiomatic definitions to get around component selection limitations with ProZ . One and only one initialisation operation with no inputs has also be defined to meet ProZ initialisation requirements.\newline
\newline
Promotion was adopted for the model to split concerns between how we change character state and then record that in the global state increasing readability. Additional functions were defined to help improve the readability of output operators as well.

\begin{zed}
[Multiverse, Timeline, Character] \\
\end{zed}
\begin{zed}
LivingStatus ::= dead | alive \\
BeingType ::= human | immortal | superhero \\
CharacterLocation == Timeline \cross Character \\
\end{zed}

\begin{axdef}
timeline: CharacterLocation \fun Timeline \\
\where
\forall x: Timeline; y: Character @ \\
\t1 timeline(x, y) = x \\
\end{axdef}

\begin{axdef}
character: CharacterLocation \fun Character \\
\where
\forall x: Timeline; y: Character @ \\
\t1 character(x, y) = y \\
\end{axdef} 

\begin{schema}{CharacterState}
being: BeingType \\
status: LivingStatus \\
lives: 0 \upto 3  \\ 
revives: 1 \upto 3 \\ 
\where
being = immortal \implies (revives, lives, status) = (1, 1, alive) \\
being = human \implies revives = 1 \\
lives \leq revives \\
\end{schema}

\begin{schema}{Megaverse}
multiverses: \power Multiverse \\
characters: \power Character \\
timelines: \power Timeline \\ 
multiverseTimelines: Multiverse \rel Timeline \\
characterStates: CharacterLocation \pfun CharacterState \\
\where
\dom multiverseTimelines \subseteq multiverses \\ 
\ran multiverseTimelines \subseteq timelines \\ 
\forall t: timelines @ \#\{ m: multiverses | (m, t) \in multiverseTimelines \} \leq 1 \\
\{ p: \dom characterStates @ timeline(p) \} \subseteq timelines \\
\{ p: \dom characterStates @ character(p) \} \subseteq characters \\
\end{schema}

\begin{schema}{Init}
Megaverse \\ 
\where
multiverses = \emptyset \\
characters = \emptyset \\ 
timelines = \emptyset \\
multiverseTimelines = \emptyset \\
characterStates = \emptyset \\
\end{schema}

\begin{schema}{CreateMultiverse}
\Delta Megaverse \\
m?: Multiverse \\
\where
m? \notin multiverses \\
characters' = characters \\
timelines' = timelines \\
multiverseTimelines' = multiverseTimelines \\
multiverses' = multiverses \cup \{ m? \} \\ 
characterStates' = characterStates \\
\end{schema}

\begin{schema}{ResetMultiverse}
\Delta Megaverse \\
m?: Multiverse \\
\where
m? \in multiverses \\
multiverseTimelines \limg \{ m? \} \rimg \neq \emptyset \\
characters' = characters \\
timelines' = timelines \setminus multiverseTimelines \limg \{ m? \} \rimg   \\
multiverses' = multiverses \\
multiverseTimelines' = \{ m? \} \ndres multiverseTimelines \\
characterStates' = \{ p: \dom characterStates | \\
\t1 timeline(p) \in multiverseTimelines \limg \{ m? \} \rimg 
\} \ndres characterStates \\
\end{schema} 

\begin{schema}{DeleteMultiverse}
\Delta Megaverse \\
m?: Multiverse \\
\where
m? \in multiverses \\
characters' = characters \\
timelines' = timelines \setminus multiverseTimelines \limg \{ m? \} \rimg   \\
multiverses' = multiverses \setminus \{ m? \} \\
multiverseTimelines' = \{ m? \} \ndres multiverseTimelines \\
characterStates' = \{ p: \dom characterStates | timeline(p) \\
\t1 \in multiverseTimelines \limg \{ m? \} \rimg \} \ndres characterStates \\
\end{schema} 

\begin{schema}{CreateTimeline}
\Delta Megaverse \\
t?: Timeline \\
\where
t? \notin timelines \\ 
characters' = characters \\ 
multiverses' = multiverses \\
characterStates' = characterStates \\ 
timelines' = timelines \cup \{ t? \} \\
multiverseTimelines' = multiverseTimelines \\
characterStates' = characterStates \\
\end{schema}

\begin{schema}{AddTimelineToMultiverse}
\Delta Megaverse \\
t?: Timeline \\
m?: Multiverse \\
\where
(m?, t?) \notin multiverseTimelines \\ 
characters' = characters \\ 
multiverses' = multiverses \\
characterStates' = characterStates \\ 
timelines' = timelines \\
multiverseTimelines' = multiverseTimelines \cup \{ m? \mapsto t? \} \\
characterStates' = characterStates \\
\end{schema}

\begin{schema}{RemoveTimelineFromMultiverse}
\Delta Megaverse \\
t?: Timeline \\
m?: Multiverse \\
\where
(m?, t?) \in multiverseTimelines \\ 
characters' = characters \\ 
multiverses' = multiverses \\
characterStates' = characterStates \\ 
timelines' = timelines \\
multiverseTimelines' = multiverseTimelines \setminus \{ m? \mapsto t? \} \\
characterStates' = characterStates \\
\end{schema}

\begin{schema}{MoveTimeline}
\Delta Megaverse \\
t?: Timeline \\
s?: Multiverse \\
d?: Multiverse \\
\where
s? \neq d? \\
\{ s?, d? \} \subseteq multiverses \\
s? \mapsto t? \in multiverseTimelines \\
characters' = characters \\ 
multiverses' = multiverses \\
characterStates' = characterStates \\
timelines' = timelines \\ 
multiverseTimelines' = multiverseTimelines \setminus \{ s? \mapsto t? \} \cup \{ d? \mapsto t? \} \\
\end{schema}

\begin{schema}{ResetTimeline}
\Delta Megaverse \\
t?: Timeline \\
\where
t? \in timelines \\
\{ p: \dom characterStates | t? = timeline(p) \} \neq \emptyset \\
characters' = characters \\ 
multiverses' = multiverses \\
characterStates' = \{ p: \dom characterStates | timeline(p) = t? \} \ndres characterStates \\
timelines' = timelines \\
multiverseTimelines' = multiverseTimelines \nrres \{ t? \} \\
\end{schema}

\begin{schema}{DeleteTimeline}
\Delta Megaverse \\
t?: Timeline \\
\where
t? \in timelines \\
characters' = characters \\ 
multiverses' = multiverses \\
characterStates' = \{ p: \dom characterStates | timeline(p) = t? \} \ndres characterStates \\
timelines' = timelines \setminus \{ t? \} \\
multiverseTimelines' = multiverseTimelines \nrres \{ t? \} \\
\end{schema}

\begin{schema}{CreateCharacter}
\Delta Megaverse \\
c?: Character \\
\where 
c? \notin characters \\
characters' = characters \cup \{ c? \} \\
multiverses' = multiverses \\
characterStates' = characterStates \\
timelines' = timelines \\
multiverseTimelines' = multiverseTimelines \\
\end{schema}

\begin{schema}{DeleteCharacter}
\Delta Megaverse \\
c?: Character \\
\where 
c? \in characters \\
characters' = characters \setminus \{ c? \} \\
multiverses' = multiverses \\
characterStates' = \{ p: \dom characterStates | character(p) = c? \} \ndres characterStates \\
timelines' = timelines \\
multiverseTimelines' = multiverseTimelines \\
\end{schema}

\begin{schema}{AddCharacterToTimeline}
\Delta Megaverse \\
m?: Multiverse \\
t?: Timeline \\
c?: Character \\
s?: CharacterState \\ 
\where
c? \in characters \\
(s?).status = alive \\
(s?).lives \neq 0 \\
(m? \mapsto t?) \in multiverseTimelines \\
(t? \mapsto c?) \notin \dom characterStates  \\
multiverses' = multiverses \\
characters' = characters \\
characterStates' = characterStates \cup \{ (t? \mapsto c?) \mapsto s? \} \\ 
timelines' = timelines \\
multiverseTimelines' = multiverseTimelines \\
\end{schema}

\begin{schema}{RemoveCharacterFromTimeline}
\Delta Megaverse \\
m?: Multiverse \\
t?: Timeline \\
c?: Character \\
\where
(m? \mapsto t?) \in multiverseTimelines \\
(t? \mapsto c?) \in \dom characterStates  \\
multiverses' = multiverses \\
characters' = characters \\
characterStates' = \{ (t? \mapsto c?) \} \ndres characterStates \\ 
timelines' = timelines \\
multiverseTimelines' = multiverseTimelines \\
\end{schema}

\begin{schema}{KillCharacterLocal} 
\Delta CharacterState \\
\where
being \neq immortal \\
status = alive \\
lives' = lives - 1 \\
revives' = revives \\
status' = dead \\
being' = being \\
\end{schema}

\begin{schema}{ReanimateCharacterLocal}
\Delta CharacterState \\
\where
status = dead \\
lives > 0 \\
being = superhero \\
revives' = revives \\
lives' = lives \\
being' = being \\
status' = alive \\
\end{schema}

\begin{schema}{ChangeToHumanLocal}
\Delta CharacterState \\
\where
being = immortal \\
lives' = 1 \\
revives' = 1 \\
being' = human \\
status' = alive \\
\end{schema}

\begin{schema}{ChangeToSuperheroLocal}
\Delta CharacterState \\
l?: 2 \upto 3 \\
\where
status = alive \\
being = human \\
lives' = l? \\
revives' = l? \\
status' = status \\
being' = superhero \\
\end{schema}

\begin{schema}{PromoteCharacterState}
\Delta Megaverse \\
\Delta CharacterState \\
c?: Character \\
t?: Timeline \\ 
\where 
(t?, c?) \in \dom characterStates \\ 
\theta CharacterState = characterStates(t?, c?) \\
characterStates' = characterStates \oplus \{ (t?, c?) \mapsto \theta CharacterState' \}   \\
multiverses' = multiverses \\
characters' = characters \\
timelines' = timelines \\
multiverseTimelines' = multiverseTimelines \\
\end{schema}

\begin{zed}
GlobalKillCharacter \defs \\
\t1 \exists \Delta CharacterState @ KillCharacterLocal \land PromoteCharacterState \\ 
\end{zed}

\begin{zed}
GlobalReanimateCharacter \defs \\
\t1 \exists \Delta CharacterState @ ReanimateCharacterLocal \land PromoteCharacterState \\
\end{zed}

\begin{zed}
GlobalChangeToHuman \defs \\ 
\t1 \exists \Delta CharacterState @ ChangeToHumanLocal \land PromoteCharacterState \\
\end{zed}

\begin{zed}
GlobalChangeToSuperhero \defs \\
\t1 \exists \Delta CharacterState @ ChangeToSuperheroLocal \land PromoteCharacterState \\
\end{zed}

\begin{axdef}
beingAndStatus: CharacterState \fun BeingType \cross LivingStatus \\
\where
\forall state: CharacterState @ \\
\t1 beingAndStatus(state) = (state.being, state.status) \\
\end{axdef}

\begin{schema}{LivingStatusForCharactersInTimeline}
\Xi Megaverse \\
t?: Timeline \\ 
r!: \power (Character \cross LivingStatus) \\
\where 
t? \in timelines \\
r! = \{ c: characters | (t?, c) \in \dom characterStates @ \\
\t1 c \mapsto (characterStates(t?,c)).status
\} \\ 
\end{schema}

\begin{schema}{CharacterStatusAcrossAllTimelines}
\Xi Megaverse \\
c?: Character \\ 
r!: \power (Timeline \cross (BeingType \cross LivingStatus)) \\
\where 
c? \in characters \\
r! = \{ t: timelines | (t, c?) \in \dom characterStates @  \\
\t1 t \mapsto beingAndStatus(characterStates((t, c?)))   
\} \\ 
\end{schema}

\pagebreak

\subsection*{Part B}

\begin{verbatim}
MACHINE Megaverse
   /* definition split up below for readability */
END
\end{verbatim}

\begin{verbatim}
SETS 
  MULTIVERSE;
  TIMELINE;
  CHARACTER;
  LIVING_STATUS = { dead, alive };
  BEING_TYPE = { human, immortal, superhero }
DEFINITIONS
  CHARACTER_STATE == 
    struct(being: BEING_TYPE, status: LIVING_STATUS, lives: 0..3, revives: 1..3);
  CHARACTER_LOCATION == TIMELINE * CHARACTER;
  BEING_AND_STATUS(s) == (s'being, s'status) 
VARIABLES
  multiverses, timelines, multiverseTimelines, characters, characterStates
INVARIANT
  multiverses: POW(MULTIVERSE)
  &
  timelines: POW(TIMELINE)
  & 
  characters: POW(CHARACTER)
  &
  multiverseTimelines: MULTIVERSE <-> TIMELINE
  &
  characterStates: CHARACTER_LOCATION +-> CHARACTER_STATE
  &
  dom(multiverseTimelines) <: multiverses
  &
  ran(multiverseTimelines) <: timelines
  &
  { p•p: dom(characterStates) | prj1(p) } <: timelines
  &
  { p•p: dom(characterStates) | prj2(p) } <: characters
  &
  !(s).(s: ran(characterStates) => s'lives <= s'revives)
  &
  !(s).(s: ran(characterStates) & s'being = human => s'revives = 1)
  &
  !(s).(s: ran(characterStates) & s'being = immortal =>
      (s'revives, s'lives, s'status) = (1, 1, alive))
  &
  !(t).(t: timelines => 
      card({ m•m: multiverses & (m,t): multiverseTimelines | (m,t) }) <= 1)
\end{verbatim}

\begin{verbatim}
INITIALISATION
  multiverses := {}
  ||
  timelines := {}
  ||
  characters := {}
  ||
  multiverseTimelines := {}
  ||
  characterStates := {}
\end{verbatim}

\begin{verbatim}
OPERATIONS
  CreateMultiverse(m) =
    PRE
      m: MULTIVERSE 
      &
      m/: multiverses
    THEN
      multiverses := multiverses \/ { m }
    END;
  ResetMultiverse(m) =
    PRE
      m: MULTIVERSE 
      &
      m: multiverses
      &
      multiverseTimelines[{ m }] /= {}
    THEN
      timelines := timelines \ multiverseTimelines[{ m }];
      characterStates := { p|p: dom(characterStates) & 
          prj1(p): multiverseTimelines[{ m }] } <<| characterStates; 
      multiverseTimelines := { m } <<| multiverseTimelines
    END;
  DeleteMultiverse(m) =
    PRE
      m: MULTIVERSE
      &
      m: multiverses
    THEN
      multiverses := multiverses \ { m };
      timelines := timelines \ multiverseTimelines[{ m }];
      characterStates := { p|p: dom(characterStates) & 
        prj1(p): multiverseTimelines[{ m }] } <<| characterStates; 
      multiverseTimelines := { m } <<| multiverseTimelines
    END;
\end{verbatim}
\pagebreak
\begin{verbatim}
  CreateTimeline(t) =
    PRE
      t: TIMELINE
      &
      t/: timelines
    THEN
      timelines := timelines \/ { t }
    END;
  ResetTimeline(t) =
    PRE
      t: TIMELINE
      &
      t: timelines
      &
      { p|p: dom(characterStates) & t = prj1(p) } /= {}
    THEN
      multiverseTimelines := multiverseTimelines |>> { t }; 
      characterStates := { p|p: dom(characterStates) & prj1(p) = t }
          <<| characterStates 
    END;
  DeleteTimeline(t) =
    PRE
      t: TIMELINE
      &
      t: timelines
    THEN
      multiverseTimelines := multiverseTimelines |>> { t }; 
      timelines := timelines \ { t };
      characterStates := { p|p: dom(characterStates) & prj1(p) = t } 
          <<| characterStates
    END;
  AddTimelineToMultiverse(t, m) =
    PRE
      t: TIMELINE
      &
      m: MULTIVERSE
      &
      t: timelines
      &
      m: multiverses
      &
      t /: ran(multiverseTimelines)
    THEN
      multiverseTimelines := multiverseTimelines \/ { m |-> t }
    END;
\end{verbatim}
\pagebreak
\begin{verbatim}
  RemoveTimelineFromMultiverse(t, m) =
    PRE
      t: TIMELINE
      &
      m: MULTIVERSE
      &
      t: timelines
      &
      m: multiverses
      &
      m |-> t : multiverseTimelines
    THEN
      multiverseTimelines := multiverseTimelines \ { m |-> t }
    END;
  MoveTimeline(t, s, d) =
    PRE
      t: TIMELINE
      & 
      s: MULTIVERSE
      &
      d: MULTIVERSE
      &
      t: timelines
      &
      s /= d
      &
      s |-> t : multiverseTimelines
      &
      {s, d} <: multiverses
    THEN
      multiverseTimelines := multiverseTimelines \ { s |-> t } \/ { d |-> t }
    END;
  CreateCharacter(c) =
    PRE
      c: CHARACTER 
      &
      c/: characters
    THEN
      characters := characters \/ { c }
    END;
\end{verbatim}
\pagebreak
\begin{verbatim}
  DeleteCharacter(c) =
    PRE
      c: CHARACTER
      &
      c: characters
    THEN
      characters := characters \ { c };
      characterStates := { p•p: dom(characterStates) & prj2(p) = c | 
        (prj1(p), c) } <<| characterStates 
    END;
  AddCharacterToTimeline(m, t, c, s) =
    PRE
      m: MULTIVERSE
      &
      t: TIMELINE
      &
      c: CHARACTER
      &
      s: CHARACTER_STATE
      &
      c: characters
      &
      s'status = alive
      & 
      s'lives /= 0 
      &
      s'lives <= s'revives
      &
      (m |-> t): multiverseTimelines  
      &
      (t |-> c)/: dom(characterStates)
      &
      (s'being = human => s'revives = 1)
      &
      (s'being = immortal => (s'revives, s'lives, s'status) = (1, 1, alive))
    THEN
      characterStates := characterStates \/ { (t |-> c) |-> s }
    END;
\end{verbatim}
\pagebreak
\begin{verbatim}
  RemoveCharacterFromTimeline(m, t, c) =
    PRE
      m: MULTIVERSE
      &
      t: TIMELINE
      &
      c: CHARACTER 
      &
      (m |-> t): multiverseTimelines
      &
      (t |-> c): dom(characterStates)
    THEN
      characterStates := { t |-> c } <<| characterStates
    END;
  ChangeToHuman(t,c) =
    PRE
      t: TIMELINE 
      &
      c: CHARACTER
      &
      (t,c): dom(characterStates)
      &
      characterStates((t,c))'being = immortal
    THEN
      characterStates(t,c) := rec(being: human, status: alive, lives: 1, revives: 1)
    END;
  ChangeToSuperhero(t,c,l) =
    PRE
      t: TIMELINE 
      &
      c: CHARACTER
      &
      l: 2..3
      &
      (t,c): dom(characterStates)
      &
      characterStates((t,c))'being = human
      & 
      characterStates((t,c))'status = alive
    THEN
      characterStates(t,c) := rec(being: superhero, status: alive, lives: l, revives: l)
    END;
\end{verbatim}
\pagebreak
\begin{verbatim}
  KillCharacter(c, t) =
    PRE
      t: TIMELINE 
      &
      c: CHARACTER
      &
      (t,c): dom(characterStates)
      &
      characterStates((t,c))'status = alive
      &
      characterStates((t, c))'being: { human, superhero }
    THEN
      LET s BE s = characterStates(t,c) IN 
        characterStates(t,c) := 
          rec(being: s'being, status: dead, lives: s'lives - 1, revives: s'revives)
      END
    END;
  ReanimateCharacter(t, c) =
    PRE
      t: TIMELINE 
      &
      c: CHARACTER
      &
      (t,c): dom(characterStates)
      &
      characterStates((t,c))'status = dead
      &
      characterStates((t,c))'lives > 0
    THEN
      LET s BE s = characterStates(t,c) IN 
        characterStates(t,c) := 
          rec(being: s'being, status: alive, lives: s'lives, revives: s'revives)
      END
    END;
  r <-- LivingStatusForCharactersInTimeline(t) = 
    PRE
      t: TIMELINE
      &
      t: timelines
      &
      r: POW(CHARACTER * LIVING_STATUS)
    THEN 
      r := { c•c: characters & (t,c): dom(characterStates) | 
        c |-> characterStates(t, c)'status }
    END;
\end{verbatim}
\pagebreak
\begin{verbatim}
  r <-- CharacterStatusAcrossAllTimelines(c) =
    PRE
      c: CHARACTER
      &
      c: characters
      &
      r: POW(TIMELINE * (BEING_TYPE * LIVING_STATUS))
    THEN
      r := { t•t: timelines & (t,c): dom(characterStates) | 
        t |-> BEING_AND_STATUS(characterStates(t,c)) }
    END
\end{verbatim}

\pagebreak

\subsection*{Part C}

For extending the model I thought it would make sense to expand the detail we capture in the CharacterState schema. We currently have a BeingType free type but it’s rather overloaded and expresses multiple things together that is rather limiting. What if we wanted a non-human character to be capable of being a superhero? How might we represent a character who has powers but may be a villain in the story? For this I’ve created the below replacement free types.

\begin{zed}
Mortality ::= mortal | immortal \\
Alignment ::= neutral | hero | villan \\
Power ::= enhanced | mutant | magic | cosmic \\
Species ::= human | alien | titan | android \\
\end{zed}

\hspace{-0.64cm} This introduces the first state-based problem, if human is now captured as a species how do we capture when the human has powers or not. To solve this problem I’ve created the below free type with a constructor that allows us to express when a character has powers or not.

\begin{zed}
Powers ::= none | some \ldata Power \rdata \\
\end{zed}

\hspace{-0.64cm} Which means we can now add these details to the CharacterState schema, which has knock on effects. Updating our constraints to deal with these new details, using the new mortality/powers details in place of the existing being type attribute.

\begin{schema}{CharacterState}
alignment: Alignment \\
mortality: Mortality \\
powers: Powers \\
species: Species \\
lives: 0 \upto 3  \\ 
revives: 1 \upto 3 \\ 
status: LivingStatus \\
\where
mortality = immortal \implies (revives, lives, status) = (1, 1, alive) \\
powers = none \implies revives = 1 \\ 
lives \leq revives \\
\end{schema}

\hspace{-0.64cm} This creates another problem, the become human and become superhero operators don’t make much sense with these new attributes. Should an alien immortal become human if they give up their immortality? Can only humans become superheroes? To resolve it made sense to redefine them as a gain powers operator and give up immortality operator that allows our new attributes to be expressed and expand the range of possible states in the model. \\
\newline
The gain powers operator accepts alignment and power as inputs so they can be declared when a character gains a power. The giving up immortality operatory now longer changes the characters species but its preconditions have been updated to look at the new mortality attribute. That precondition change was also needed for the kill character operator and the resurrecting character operator now needed to move to use the powers attribute. Of course, all these operators now needed to detail how the new properties change when the operator is executed. \\

\begin{schema}{KillCharacterLocal} 
\Delta CharacterState \\
\where
status = alive \\
mortality \neq immortal \\
alignment' = alignment \\
mortality' = mortality \\
powers' = powers \\ 
species' = species \\
lives' = lives - 1 \\
revives' = revives \\
status' = dead \\
\end{schema}

\begin{schema}{ResurrectCharacterLocal}
\Delta CharacterState \\
\where
lives > 0 \\
status = dead \\
powers \neq none \\ 
alignment' = alignment \\
mortality' = mortality \\
powers' = powers \\ 
species' = species \\
lives' = lives \\
revives' = revives \\
status' = alive \\
\end{schema}

\begin{schema}{GainPowersLocal}
\Delta CharacterState \\
p?: Power \\
a?: Alignment \\
l?: 2 \upto 3 \\
\where
status = alive \\ 
species = human \\
powers = none \\
alignment' = a? \\
mortality' = mortality \\
powers' = some(p?) \\ 
species' = species \\
lives' = l? \\
revives' = l? \\
status' = status \\
\end{schema}

\begin{schema}{GiveUpImmortalityLocal}
\Delta CharacterState \\
\where
mortality = immortal \\
alignment' = alignment \\
mortality' = mortality \\
powers' = powers \\ 
species' = species \\
lives' = 1 \\
revives' = 1 \\
status' = alive \\
\end{schema}

\hspace{-0.64cm} Finally with our redefined and updated operators we need to define them in the context of the global state schema so these local operations can be promoted from CharacterState to the Megaverse. 

\begin{zed}
GlobalGiveUpImmortality \defs \exists \Delta CharacterState @ GiveUpImmortalityLocal \land PromoteCharacterState \\
GlobalResurrectCharacter \defs \exists \Delta CharacterState @ ResurrectCharacterLocal \land PromoteCharacterState \\
GlobalKillCharacter \defs \exists \Delta CharacterState @ KillCharacterLocal \land PromoteCharacterState \\ 
GlobalGainPowers \defs \exists \Delta CharacterState @ GainPowersLocal \land PromoteCharacterState \\
\end{zed}

\hspace{-0.64cm} Adding these extensions to the B model starts with adding new sets for our new free type definitions. But for the powers free type that needs to express a full free type in B as it can’t be expressed as a set and following that through by adding the new attributes to the CharacterState struct.

\begin{verbatim}
SETS 
  /* ... */
  MORTALITY = { mortal, immortal };
  ALIGNMENT = { neutral, hero, villan };
  POWER = { enhanced, mutant, magic, cosmic };
  SPECIES = { human, alien, titan, android }
\end{verbatim}
\begin{verbatim}
FREETYPES
  POWERS = NONE, SOME(POWER)
\end{verbatim}
\begin{verbatim}
DEFINITIONS
  CHARACTER_STATE == 
    struct(alignment: ALIGNMENT, mortality: MORTALITY, powers: POWERS, 
      species: SPECIES, lives: 0..3, revives: 1..3, status: LIVING_STATUS);
\end{verbatim}

\hspace{-0.64cm} With the new attributes stored, we then need to update the corresponding invariants to make use of the new attributes. This then extends to updating the pre-existing preconditions for the add character to timeline operation to make sure we can’t add a character state that violates the invariants we just corrected.

\begin{verbatim}
INVARIANT
  /* ... */
  !(s).(s: ran(characterStates) & s'powers = NONE => s'revives = 1)
  &
  !(s).(s: ran(characterStates) & s'mortality = immortal 
      => (s'revives, s'lives, s'status) = (1, 1, alive))
\end{verbatim}

\begin{verbatim}
AddCharacterToTimeline(m, t, c, s) =
  PRE
    /* ... */
    (s'powers /= NONE => s'revives = 1)
    &
    (s'mortality = immortal => (s'revives, s'lives, s'status) = (1, 1, alive))
\end{verbatim}

\hspace{-0.64cm} Replicating the gain powers and give up immortality operators in B translates very similarly to what we did in the Z. Preconditions are updated and new input parameters for the gain powers operation. 

\begin{verbatim}
KillCharacter(c, t) =
  PRE
    t: TIMELINE 
    &
    c: CHARACTER
    &
    (t,c): dom(characterStates)
    &
    characterStates((t,c))'status = alive
    &
    characterStates((t, c))'mortality = mortal
  THEN
    LET s BE s = characterStates(t,c) IN 
      characterStates(t,c) := 
        rec(alignment: s'alignment, mortality: s'mortality, powers: s'powers, 
          species: s'species, lives: s'lives - 1, revives: s'revives, status: dead)
    END
  END;
\end{verbatim}
\begin{verbatim}
ResurrectCharacter(t, c) =
  PRE
    t: TIMELINE 
    &
    c: CHARACTER
    &
    (t,c): dom(characterStates)
    &
    characterStates((t,c))'status = dead
    &
    characterStates((t,c))'lives > 0
  THEN
    LET s BE s = characterStates(t,c) IN 
      characterStates(t,c) := 
        rec(alignment: s'alignment, mortality: s'mortality, powers: s'powers, 
          species: s'species, lives: s'lives, revives: s'revives, status: alive)
    END
  END;
\end{verbatim}
\pagebreak
\begin{verbatim}
  GainPowers(t,c,l,p,a) =
    PRE
      t: TIMELINE 
      &
      c: CHARACTER
      &
      l: 2..3
      &
      p: POWER
      &
      a: ALIGNMENT
      &
      (t,c): dom(characterStates)
      &
      characterStates((t,c))'powers = NONE
      & 
      characterStates((t,c))'status = alive
    THEN
      LET s BE s = characterStates(t,c) IN
        characterStates(t,c) := 
          rec(alignment: a, mortality: s'mortality, powers: SOME(p), 
            species: s'species, lives: l, revives: l, status: alive)
      END
    END;

  GiveUpImmortality(t,c) =
    PRE
      t: TIMELINE 
      &
      c: CHARACTER
      &
      (t,c): dom(characterStates)
      &
      characterStates((t,c))'mortality = immortal
    THEN
      LET s BE s = characterStates(t,c) IN
        characterStates(t,c) := 
          rec(alignment: s'alignment, mortality: s'mortality, powers: s'powers, 
            species: s'species, lives: 1, revives: 1, status: alive)
      END
    END;
\end{verbatim}
\pagebreak

\section*{Q2}

\subsection*{Introduction}

Software based systems can be complex and are often at a scope that is too large for any one developer to fully comprehend. State based modelling is a powerful technique that can help bring order to overloaded complexity through description, refinement and proof that helps establish the fundamental properties and behaviours of stateful systems.
\newline \newline
To illustrate the benefits and limitations of state-based modelling in general and the relationship between Z and B, a basic trading system model has been designed for the capture and storage of securities lending trades. These are transactions between parties who wish to borrow or loan a given security in return for a nominal fee and are a critical underpinning of the financial system.
\newline \newline
By leveraging this model and with additional support through referenced case studies and papers we aim to illustrate the benefits and limitations of state-based modelling. Furthermore, by navigating from a Z to a B, representation of our trading system, we aim to highlight the shared heritage of both Z notation and the B-Method and how their bespoke differences provide unique opportunities to be leveraged in a variety of scenarios.

\subsection*{State based modelling}

State-based modelling is a technique for describing systems by defining their state, the constraints and guards that apply to that state, and the transitions that states can undergo via operations. Therefore resulting in a comprehensive and structured definition of the system's model and associated behaviours, capabilities, and limitations.
\newline \newline
By formally defining state normally through mathematical notation the aim is to remove ambiguities and systematically prove model correctness and consistency as it transitions through various states.  Capturing and resolving errors in model definitions before implementation in a software system improves system safety.
\newline \newline
Various methods and tools exist with their own sets of capabilities. Two more popular methods in use in academia and industry are Z notation and the B-Method. They share a common heritage, and both can be leveraged for writing formal specifications. We will use both to highlight the benefits and limitations of each method as well as the common attributes that apply to state-based modelling in general.

\subsection*{Z notation}

Z notation is a formal specification language based on set theory and first-order logic, designed to describe state and associated behaviours with mathematical precision. Initially conceived by Jean Raymond Abrial at the University of Grenoble it was refined in collaboration with the Oxford University Computing Library. Since then, it has found use in industry at IBM, Rolls-Royce and British Aerospace for verifying system specifications.
\newline \newline
At the heart of Z is the mathematical language that allows practitioners to work with sets, bags, sequences, types, functions, relations and logic that can then be collected and organised into schemas where declarations are made alongside constraints. These schemas form the basis of how Z can be leveraged to describe the model of a system, not just from describing the state in the system, types of properties and the constraints bound to them but from how the system’s state can transition through local and global operations.
\newline \newline
Z is, however, not intended for the description and specification of nonfunctional properties such as usability, performance, security, efficiency, or reliability. Implementation level details on how logic could be concurrently scheduled, or state storage optimised in memory are not the domain of Z. Additional tools can be leveraged to fill that gap and complement the state-based modelling capabilities found in Z.

\subsubsection*{Trading System Model}

The trading system model in Z starts with some helper types. An ‘Identifier’ type for use as a readable unique identifier for trades. A ‘Date’ type has also been created, represented a natural number for easy date comparisons.  Finally, a ‘DateOption’ free type that allows us to express an end date for our trades in an optional manner.

\begin{zed}
Identifier == Prefix \cross \{ d: \nat | d \geq 1 \land d \leq 9 \} \\ 
Date == \{ d: \nat | d \geq 20250101 \land d \leq 20991231 \} \\
DateOption ::= None | Some \ldata Date \rdata \\
\end{zed}

\hspace{-0.62cm} I've also defined the following free types: Security, Trader and Counterparty that represents the type of security being traded, the trader themselves and the counterparty the trade is with for brevity.  Direction, Status and Prefix have been defined as they are suitable representations for the properties they will represent.

\begin{zed}
Counterparty ::= Citadel | Millennium  \\
Direction ::= Borrow | Loan \\
Security ::= Bond | Stock \\
Status ::= Open | Closed \\  
Trader ::= Joe | John | Jane | Joan \\
Prefix ::= T | U | S | V \\
\end{zed}

\hspace{-0.62cm} Which now allows us to define our Trade schema which will define our local state, capturing the identifier of the trade and its associated properties. A maximum quantity size 100,000 has been expressed through our constraints in additional adding a guard that when a trade has ended it is marked as Closed. 

\begin{schema}{Trade}
identifier: Identifier \\
direction: Direction \\ 
security: Security \\
quantity: \nat_1 \\
startDate: Date \\
endDate: OptionalDate \\ 
counterparty: Counterparty \\
status: Status \\
owner: Trader \\
\where
status = Closed \iff endDate \neq None \\
quantity \leq 100000 \\
\end{schema}

\hspace{-0.7cm} Trades are stored in a global state the ‘TradingSystem’ schema which keeps a record of all trades and users who can book them. Critically we introduce a sensible constraint that users cannot be removed from the system while they have Open trades and we have provided a sensible initialisation operation for the system.

\begin{schema}{TradingSystem}
trades: Identifier \pfun Trade \\
users: \power Trader \\
\where
\forall t: \ran trades | t.status = Open @ t.owner \in users \\ 
\end{schema}

\begin{schema}{Init}
TradingSystem \\
\where
trades = \emptyset \\
users = \emptyset \\
\end{schema}

\hspace{-0.7cm} This brings us onto operations for creating and closing trades starting with ‘CreateTrade’ that enables trades booked to be capture in the system providing they aren’t there already. The operation definition is more verbose than expected due to ProZ’s lack of support for bindings an the large number of properties on the’Trade’.

\begin{schema}{CreateTrade}
\Delta TradingSystem \\
t?: Trade \\
\where 
(t?).identifier \notin \dom trades \\
(t?).owner \in users \\
(t?).status = Open \\ 
(t?).endDate = None \\
trades' = trades \cup \{ (t?).identifier \mapsto t? \} \\
users' = users \\
\end{schema}

Our other operation is ‘CloseTrade’ which allows the trader to terminate the trade at a date after the start date. This operation has be defined using promotion as one can imagine multiple operations being applicable to an Open trade and this technique provides a clean dividing line between the local state changes for Trade and then how that should propagate to the ‘TradingSystem’.

\begin{schema}{CloseTrade}
\Delta TradingSystem \\
i?: Identifier \\
d?: Date \\ 
\where
(i?) \in \dom trades \\
(trades(i?)).status = Open \\ 
(trades(i?)).startDate < d? \\
users' = users \\
(\{ i? \} \ndres trades') = \{ i? \} \ndres trades \\
(trades'(i?)).startDate = (trades(i?)).startDate \\
(trades'(i?)).identifier = (trades(i?)).identifier \\
(trades'(i?)).direction = (trades(i?)).direction \\
(trades'(i?)).counterparty = (trades(i?)).counterparty \\
(trades'(i?)).security = (trades(i?)).security \\
(trades'(i?)).quantity = (trades(i?)).quantity \\
(trades'(i?)).owner = (trades(i?)).owner \\
(trades'(i?)).endDate = Some(d?) \\ 
(trades'(i?)).status = Closed \\
\end{schema}

\hspace{-0.7cm} Finally, we have two operations that allow traders to be added to the system and removed. We provide constraints to make sure the user has not been added yet or does exist to be removed and let the constraint for ‘TradingSystem’ enforce the constraint on when a trader can be removed from the system. 

\begin{schema}{AddTrader}
\Delta TradingSystem \\
u?: Trader \\ 
\where
u? \notin users \\
users' = users \cup \{ u? \} \\
trades' = trades \\
\end{schema}

\begin{schema}{RemoveTrader}
\Delta TradingSystem \\
u?: Trader \\ 
\where
u? \in users \\
users' = users \setminus \{ u? \} \\
trades' = trades \\
\end{schema}

\subsubsection*{B}

B is another formal method originally developed by Abrial that like Z is based on set theory and first-order logic. The method was developed to address the lack of support in Z for a systematic refinement process to transition to an implementable system. Leveraged as part of the B-Method it saw industry adoption most notably at Alstom and Siemens the design and specification of railway systems.  
\newline \newline
B has a lot in common with Z, which will in part been due to Abrial playing a large role in the development of both methods. But at the heart of B specifications are abstract machines where the practitioner can define much of what we could define in Z through defining sets, constants, properties, variables, invariants, initialisation and operations for a given abstract state machine.
\newline \newline
The B-method like Z notation before it lacks support for nonfunctional requirements despite being closer to an implementation level description. Like Z, the B-Method is primarily a state-based modelling tool but with greater emphasis on refinement and proof processes. Describing properties outside of this domain will require working with additional tools or formal methods such as Atelier B, Event-B and TLA+ that aim to address these points.

\subsubsection*{Trading System Model}

Our trading system model in B starts with the machine definition. Because we need to express date as an option our DateOption free type in Z needs to be declared as a free type in B so we can express the constructor function.
\begin{verbatim}
MACHINE TradingSystem
  FREETYPES
    DATEOPTION = None, Some(DATE)
  ...
END
\end{verbatim}

\hspace{-0.7cm} The remaining free types are expressed as sets in our abstract state machine and mirror the definitions in our Z representation.

\begin{verbatim}
SETS 
  COUNTERPARTY = { Citadel, Millennium };
  DIRECTION = { Borrow, Loan };
  SECURITY = { Bond, Stock };
  STATUS = { Open, Closed };
  TRADER = { Joe, John, Jane, Joan };
  PREFIX = { T, U, S, V }
\end{verbatim}

\hspace{-0.7cm} Our helper types such as 'IDENTIFIER' and 'DATE' are declared as definitions. 'TRADE' has been defined as a struct mirroring the schema type we declared in Z spec.

\begin{verbatim}
DEFINITIONS
  DATE == 20250101..20991231;
  IDENTIFIER == PREFIX * (1..9);
  TRADE == struct( identifier: IDENTIFIER, direction: DIRECTION, security: SECURITY, quantity: NAT1, counterparty: COUNTERPARTY,startDate: DATE, endDate: DATEOPTION, status: STATUS, owner: TRADER )
\end{verbatim}

\hspace{-0.7cm} For capturing users of the trading system and trades booked we mirror the global state from the Z specification with a trades and users variables.

\begin{verbatim}
VARIABLES
  trades, users
INITIALISATION
  trades := {}
  ||
  users := {}
\end{verbatim}
    
\hspace{-0.7cm} The invariants for the machine are declared with trades are partial function mapping identifiers to a trade and users a set of traders. I've also imported our constriants around max quantity, trades that have an end date being marked with the correct state and finally expressing the condition that traders with open trades cannot be removed from the system.

\begin{verbatim}
INVARIANT
  trades: IDENTIFIER +-> TRADE
  &
  users: POW(TRADER)    
  &
  !(t).(t: ran(trades) => t'quantity < 100000)
  &
  !(t).(t: ran(trades) => t'status = Closed <=> t'endDate /= None)
  &
  !(t).(t: ran(trades) & t'status = Open => t'owner: users)
\end{verbatim}

\hspace{-0.7cm} Which moves us finally onto a operations. Starting with adding and removing traders and then creating and closing trades preconditions to make sure we don't invalidate the invariants we specified earlier. 

\begin{verbatim}
  OPERATIONS
    ...
  END
\end{verbatim}

\pagebreak

\hspace{-0.7cm} Something something something

\begin{verbatim}
CreateTrade(t) =
  PRE
    t: TRADE
    &
    t'identifier /: dom(trades)
    &
    t'owner: users 
    &
    t'status = Open 
    &
    t'endDate = None 
    &
    t'quantity < 100000
  THEN
    trades := trades \/ { t'identifier |-> t }
  END;
\end{verbatim}

\hspace{-0.7cm} Something something something

\begin{verbatim}
CloseTrade(i, d) =
  PRE
    i: IDENTIFIER
    &
    d: DATE
    &
    i: dom(trades)
    &
    trades(i)'startDate < d
    &
    trades(i)'status = Open
    &
    trades(i)'endDate = None
  THEN
    LET trade BE t = trades(i) IN
      trades := trades <+ { 
        i |-> rec( 
          identifier: t'identifier, 
          direction: t'direction, 
          security: t'security, 
          quantity: t'quantity, 
          startDate: t'startDate, 
          counterparty: t'counterparty, 
          owner: t'owner, 
          endDate: Some(d),
          status: Closed 
        ) 
      }
    END
  END 
\end{verbatim}

\pagebreak


\hspace{-0.7cm} Something something something

\begin{verbatim}
AddTrader(u) = 
  PRE
    u: TRADER
    &
    u/: users
  THEN
    users := users \/ {u}
END;
\end{verbatim}

\hspace{-0.7cm} Something something something

\begin{verbatim}
  RemoveTrader(u) = 
    PRE
      u: TRADER
      &
      u:users
      &
      {t | t: ran(trades) & t'owner = u & t'status = Open } = {}
    THEN
      users := users - {u}
    END;
END
\end{verbatim}

\subsection*{Benefits and Limitations}

... \pagebreak

\subsection*{Relationship between Z and B}

... \pagebreak

\subsection*{Conclusions}

... \pagebreak

\subsection*{References}

...

\end{document}
