\documentclass{article}

% Packages
\usepackage{fuzz}
\usepackage[margin=1.0in]{geometry}
\usepackage{graphicx}
\usepackage{listings}
\usepackage{color}
\definecolor{dkgreen}{rgb}{0,0.6,0}
\definecolor{gray}{rgb}{0.5,0.5,0.5}
\definecolor{mauve}{rgb}{0.58,0,0.82}
\lstset{language=SQL,
  basicstyle={\small\ttfamily},
  belowskip=3mm,
  breakatwhitespace=true,
  breaklines=true,
  classoffset=0,
  columns=flexible,
  commentstyle=\color{dkgreen},
  framexleftmargin=0.5em,
  frameshape={}{}{}{},
  keywordstyle=\color{blue},
  numbers=none,
  numberstyle=\tiny\color{gray},
  showstringspaces=false,
  stringstyle=\color{mauve},
  tabsize=3,
  xleftmargin =0em
}
\definecolor{delim}{RGB}{20,105,176}
\definecolor{numb}{RGB}{106, 109, 32}
\definecolor{string}{rgb}{0.64,0.08,0.08}

\lstdefinelanguage{json}{
    numbers=none,
    numberstyle=\small,
    frame=none,
    rulecolor=\color{black},
    showspaces=false,
    showtabs=false,
    breaklines=false,
    postbreak=\raisebox{0ex}[0ex][0ex]{\ensuremath{\color{gray}\hookrightarrow\space}},
    breakatwhitespace=true,
    basicstyle=\ttfamily\small,
    upquote=true,
    morestring=[b]",
    stringstyle=\color{string},
    literate=
     *{0}{{{\color{numb}0}}}{1}
      {1}{{{\color{numb}1}}}{1}
      {2}{{{\color{numb}2}}}{1}
      {3}{{{\color{numb}3}}}{1}
      {4}{{{\color{numb}4}}}{1}
      {5}{{{\color{numb}5}}}{1}
      {6}{{{\color{numb}6}}}{1}
      {7}{{{\color{numb}7}}}{1}
      {8}{{{\color{numb}8}}}{1}
      {9}{{{\color{numb}9}}}{1}
      {\{}{{{\color{delim}{\{}}}}{1}
      {\}}{{{\color{delim}{\}}}}}{1}
      {[}{{{\color{delim}{[}}}}{1}
      {]}{{{\color{delim}{]}}}}{1},
}
\usepackage{prooftree}

% Metadata
\title{Final Assignment}
\date{\vspace{-1.0cm}18th June 2024}

% Start
\begin{document}
\maketitle

\section*{Q1}

\subsection*{\small a)}
\textbf{Personal Information}
\newline \newline Assumptions:
\begin{itemize}
    \item I've assumed that the schema doesn't need to support a broad list of genders and therefore it's stored as an attribute on the $User$ relation rather than split into its own relation.
\end{itemize}
\begin{schema}{User}
	\underline{userId}: UserIdType \\
  emailAddress: EmailAddressType \\
  firstName: NameType \\
  lastName: NameType \\
  dateOfBirth: DateType \\
  gender: GenderType \\
  height: HeightType \\ 
  createdAt: DateTimeType \\
\end{schema}
\begin{zed}
\textcolor{red}{PK}(User) = \{ userId \} \\
\end{zed}
\newline
\textbf{Health Status Data}
\newline \newline Assumptions:
\begin{itemize}
  \item I've assumed that eventually we would capture additional attributes tied to the reading e.g. location, and therefore, it made sense to normalise the relation and store readings and metrics separately.
  \item I've also assumed blood pressure should be broken down into systolic and diastolic readings and therefore should be separate attributes.
\end{itemize}
\begin{schema}{HealthReading}
	\underline{readingId}: ReadingIdType \\
    readAt: DateTimeType \\
    userId: UserIdType \\
\end{schema}
\vspace{-0.75cm}
\begin{schema}{HealthMetrics}
    \underline{metricsId}: MetricIdType \\
    readingId: ReadingIdType \\
    weight: WeightType \\ 
    glucoseLevel: GlucoseLevelType \\
    systolicBloodPressure: BloodPressureType \\
    diastolicBloodPressure: BloodPressureType \\
\end{schema}
\begin{zed}
\textcolor{red}{PK}(HealthReading) = \{ readingId \} \\ 
\textcolor{red}{PK}(HealthMetrics) = \{ metricsId \} \\
\newline \\ 
(HealthMetrics, HealthReading) \mapsto \{ readingId \mapsto readingId \} \in \textcolor{red}{FK} \\
(HealthReading, User) \mapsto \{ userId \mapsto userId \} \in \textcolor{red}{FK} \\
\end{zed}
\newline
\textbf{Diet Data}
\newline \newline Assumptions:
\begin{itemize}
  \item That diet data is meant to be very high level, and there is no need to store a broad list of food/drink types/names. 
  \item Drinks should be captured by drink type and glasses consumed. Food should be captured by food name and quantity consumed.
  \item Calories should only be stored for food and drinks and not for meals. 
\end{itemize}
\begin{schema}{Diet}
  \underline{dietId}: DietIdType \\
  forDate: DateType \\
  userId: UserIdType \\
\end{schema}
\vspace{-0.75cm}
\begin{schema}{Meal}
	\underline{mealId}: MealIdType \\
  dietId: DietIdType \\
  mealType: MealType \\
  mealSize: SizeType \\
  mealTime: TimeType \\
\end{schema}
\vspace{-0.75cm}
\begin{schema}{Food}
	\underline{foodId}: FoodIdType \\
  dietId: DietId \\
  foodName: FoodNameType \\
  calories: CaloriesType \\
  quantity: \nat_1 \\
  consumedAt: TimeType \\
\end{schema}
\vspace{-0.75cm}
\begin{schema}{Drink}
	\underline{drinkId}: DrinkIdType \\
  dietId: DietId \\
  drinkType: DrinkType \\
  calories: CaloriesType \\
  glasses: \nat_1 \\
  consumedAt: TimeType \\
\end{schema}
\begin{zed}
\textcolor{red}{PK}(Diet) = \{ dietId \} \\
\textcolor{red}{PK}(Meal) = \{ mealId \} \\
\textcolor{red}{PK}(Food) = \{ foodId \} \\
\textcolor{red}{PK}(Drink) = \{ drinkId \} \\
\newline \\ 
(Diet, User) \mapsto \{ userId \mapsto userId \} \in \textcolor{red}{FK} \\
(Meal, Diet) \mapsto \{ dietId \mapsto dietId \} \in \textcolor{red}{FK} \\
(Food, Diet) \mapsto \{ dietId \mapsto dietId \} \in \textcolor{red}{FK} \\
(Drink, Diet) \mapsto \{ dietId \mapsto dietId \} \in \textcolor{red}{FK} \\
\end{zed}
\newline
\textbf{Gym Performance Data}
\newline \newline Assumptions:
\begin{itemize}
  \item I've assumed that we will want to support grouping weightlifting activities into a workout and that each workout may involve different exercises.
  \item I've also assumed that we will want to support a broad range of weightlifting exercise types and therefore it would make sense to have a separate relation for this.
\end{itemize}
\begin{schema}{Exercise}
	\underline{exerciseId}: ExerciseIdType \\
	name: NameType \\ 
  description: DescriptionType \\
\end{schema}
\vspace{-0.75cm}
\begin{schema}{Workout}
	\underline{workoutId}: WorkoutIdType \\
    startedAt: DateTimeType \\
    userId: UserIdType \\
\end{schema}
\vspace{-0.75cm}
\begin{schema}{WorkoutSession}
    \underline{sessionId}: SessionIdType \\
    workoutId: WorkoutIdType \\
    exerciseId: ExerciseIdType \\
    weightLevel: WeightLevelType \\
    numberOfSets: \nat_1 \\
    repetitionsPerSet: \nat_1  \\
    occuredAt: TimeType \\
\end{schema}
\begin{zed}
\textcolor{red}{PK}(Exercise) = \{ exerciseId \} \\
\textcolor{red}{PK}(Workout) = \{ workoutId \} \\
\textcolor{red}{PK}(WorkoutSession) = \{ sessionId \} \\
\newline \\ 
(WorkoutSession, Workout) \mapsto \{ workoutId \mapsto workoutId \} \in \textcolor{red}{FK} \\
(WorkoutSession, Exercise) \mapsto \{ exerciseId \mapsto exerciseId \} \in \textcolor{red}{FK} \\
\end{zed}
\newline 
\textbf{Movement Data}
\newline \newline Assumptions:
\begin{itemize}
  \item That we want to store daily and hourly movement data in separate relations to avoid overlapping meaning and more easily apply constraints to validate data consistency.
  \item That each piece of movement data should be captured for each given pace.
\end{itemize}
\begin{schema}{DailyMovement}
    \underline{dailyMovementId}: DailyMovementIdType \\
    calories: CaloriesType \\
    distance: DistanceType \\ 
    stairs: \nat \\
    steps: \nat \\
    pace: PaceType \\
    date: DateType \\
    userId: UserIdType \\
\end{schema}
\begin{schema}{HourlyMovement}
    \underline{hourlyMovementId}: HourlyMovementIdType \\
    calories: CaloriesType \\
    distance: DistanceType \\ 
    stairs: \nat \\
    steps: \nat \\
    pace: PaceType \\
    date: DateType \\
    time: TimeType \\
    userId: UserIdType \\
\end{schema}
\begin{zed}
\textcolor{red}{PK}(DailyMovement) = \{ dailyMovementId \} \\
\textcolor{red}{PK}(HourlyMovement) = \{ hourlyMovementId \} \\
\newline \\ 
(DailyMovement, User) \mapsto \{ userId \mapsto userId \} \in \textcolor{red}{FK} \\
(HourlyMovement, User) \mapsto \{ userId \mapsto userId \} \in \textcolor{red}{FK} \\
\end{zed}
\newline
\textbf{Personal Goals}
\newline \newline Assumptions:
\begin{itemize}
  \item I've assumed that goal status does not have a pending state, so it made sense to have a separate relation for tracking goal outcome as no intermediate state needed to be stored. 
  \item That we want attributes per evidence type so we can apply foreign key constraints ensuring data integrity.
  \item That any message sent to the user regarding a goal will have multiple properties so it made sense to have its own relation and open the door to multiple messages being sent for a goal.
\end{itemize}
\begin{schema}{Goal}
	\underline{goalId}: GoalIdType \\
  userId: UserIdType \\
  startOn: DateTimeType \\
  endAt: DateTimeType \\ 
  description: DescriptionType \\
  objective: ObjectiveType \\
  measurement: MeasurementType \\
  target: TargetType \\ 
\end{schema}
\vspace{-0.75cm}
\begin{schema}{GoalOutcome}
	\underline{outcomeId}: OutcomeIdType \\
  goalId: GoalIdType \\
  status: Status \\
  achievedAt: DateTimeType \\
\end{schema}
\vspace{-0.75cm}
\begin{schema}{GoalMessage}
	\underline{messageId}: MessageIdType \\
  goalId: OutcomeIdType \\
  subject: SubjectType \\
  message: MessageType \\
  sentAt: DateTimeType \\
\end{schema}
\vspace{-0.75cm}
\begin{schema}{GoalEvidence}
	\underline{evidenceId}: EvidenceIdType \\
  goalId: GoalIdType \\
  dailyMovementId: DailyMovementIdType \\
  hourlyMovementId: HourlyMovementIdType \\
  healthMetricsId: MetricIdType \\
  workoutSessionId: SessionIdType \\
\end{schema}

\begin{zed}
\textcolor{red}{PK}(Goal) = \{ goalId \}  \\
\textcolor{red}{PK}(GoalOutcome) = \{ outcomeId \}  \\
\textcolor{red}{PK}(GoalMessage) = \{ messageId \}  \\
\textcolor{red}{PK}(GoalEvidence) = \{ evidenceId \}  \\
\newline \\ 
(Goal, User) \mapsto \{ userId \mapsto userId \} \in \textcolor{red}{FK} \\
(GoalOutcome, Goal) \mapsto \{ goalId \mapsto goalId \} \in \textcolor{red}{FK} \\
(GoalMessage, Goal) \mapsto \{ goalId \mapsto goalId \} \in \textcolor{red}{FK} \\
(GoalEvidence, Goal) \mapsto \{ goalId \mapsto goalId \} \in \textcolor{red}{FK} \\
(GoalEvidence, DailyMovement) \mapsto \{ dailyMovementId \mapsto dailyMovementId \} \in \textcolor{red}{FK} \\
(GoalEvidence, HourlyMovement) \mapsto \{ hourlyMovementId \mapsto hourlyMovementId \} \in \textcolor{red}{FK} \\
(GoalEvidence, HealthMetrics) \mapsto \{ healthMetricsId \mapsto metricsId \} \in \textcolor{red}{FK} \\
(GoalEvidence, WorkoutSession) \mapsto \{ workoutSessionId \mapsto sessionId \} \in \textcolor{red}{FK} \\
\end{zed}

\subsection*{\small b)}

I designed the schema to be in 3NF and have not, at this stage, applied any denormalisation to increase performance at the expense of increasing redundancy. However, to create the schema in 3NF and validate that it is, I first had to ensure it met the rules for 1NF and 2NF. \\
\newline
I've ensured the schema is in 1NF by identifying and removing repeating groups and ensuring each attribute in each relation contains only a single value. This can be seen in the relations created for storing $Diet\ Data$; rather than our $Diet$ relation having a set of food IDs $\power FoodIdType$ or meal IDs $\power MealIdType$, I choose an alternative design. $Food$, $Drink$ and $Meal$ all have an attribute that acts as a foreign key to the daily $Diet$ entry, removing the need to store this information in the $Diet$ relation, ensuring that each attribute stored in the relation is a single atomic value, keeping the schema in 1NF. \\
\newline
With 1NF achieved across each relation in the schema, I could then move on to 2NF, where I needed to remove any partial dependencies from the schema by moving them into their own relation. This can be seen in the relations created for $Gym\ Performance\ Data$. The exercise name could have been stored as an attribute in $WorkoutSession$ and was in the initial design. However, this would have introduced a partial dependency as $exerciseName$ is not dependent on the primary key of $WorkoutSession$ $sessionId$. By moving it out into its own relation $Exercise$ with its own primary key $exerciseId$, we were able to upgrade the design to be 2NF as $exerciseName$ is now dependent on $exerciseId$ and not partially dependent on $sessionId$. \\
\newline
Finally, with 1NF and 2NF achieved, I could test the schema is in 3NF and where it wasn't, make it so, by removing transitive dependencies for non-primary key attributes. This can be seen in the relations created for tracking a user's $Personal\ Goals$. In the original design, all the main attributes were stored in a single relation $Goal$, one of which was a foreign key to the user to which the goal was tied. However, this design had transitive dependencies as attributes relating to a goal's outcome; the message sent to the user and any evidence supporting the goal were transitively dependent on $goalId$. We removed these transitive dependencies by moving related attributes into their own relations, $GoalOutcome$, $GoalEvidence$, and $GoalMessage$, ensuring the schema was in 3NF. \\

\subsection*{\small c)}
\textbf{\small{C1}}
\begin{zed}
\forall g: Goal @ \lnot (g.endOn < g.startAt) \\
\end{zed}
\textbf{\small{C2}}
\begin{zed}
\forall dm: DailyMovement @ \\
    \t1 d.calories = sum\{ h: HourlyMovement | (d.userId, d.date, d.pace) = (h.userId, h.date, h.pace) @ h.calories \} \\
    \t1 d.distance = sum\{ h: HourlyMovement | (d.userId, d.date, d.pace) = (h.userId, h.date, h.pace) @ h.distance \} \\
    \t1 d.stairs = sum\{ h: HourlyMovement | (d.userId, d.date, d.pace) = (h.userId, h.date, h.pace) @ h.stairs \} \\
    \t1 d.steps = sum\{ h: HourlyMovement | (d.userId, d.date, d.pace) = (h.userId, h.date, h.pace) @ h.steps \} \\
\end{zed}
\textbf{\small{C3}}
\begin{zed}
\forall u: User; d: DateType @ \\
\t1 \#\{ g: Goal | g.userId = u.userId \land g.startAt \leq d \land g.endOn \geq d \} \leq 3
\end{zed}
\textbf{\small{C4}}
\begin{zed}
\forall o: GoalOutcome | o.status \in \{ Success, Failure \} @ \\
\t1 \exists e: GoalEvidence | e.goalId = o.goalId @ \\
\t2 \{ h: HourlyMovement | e.hourlyMovementId = h.hourlyMovementId \} \neq \emptyset \lor \\
\t2 \{ d: DailyMovement | e.dailyMovementId = d.dailyMovementId \} \neq \emptyset \lor \\
\t2 \{ s: WorkoutSession | e.workoutSessionId = s.sessionId \} \neq \emptyset \lor \\
\t2 \{ m: HealthMetrics | e.healthMetricsId = m.metridId \} \neq \emptyset \lor \\
\end{zed}
\textbf{\small{Additional Constraints}}
\begin{itemize}
  \item For {\em Diet Data}, it would make sense to introduce constraints to ensure we never have more than one entry in $Diet$ per day for each user. It would also make sense to enforce that glasses / quantity should never be less than one for each $Food$ and $Drink$ entry.
  \item For {\em Personal Goals}, there are additional fields that should be consistent with each other. It would make sense to add a constraint to make sure a $GoalOutcome$ is only achieved during the goal period and that a $GoalMessage$ can only be stored/sent when a $GoalOutcome$ has been recorded for the goal.
  \item For {\em Movement Data}, it would make sense to introduce constraints to make sure there is never more than 1 entry, per pace, per user, per date in the $DailyMovement$ relation and there shouldn't be more than 24 entries per user, per pace, per date in the $HourlyMovement$ relation. 
  \item For {\em Gym Performance Data}, it would make sense to introduce constraints that make sure $numberOfSets$ and $repititionsPerSet$ is never less than one and that each $Workout$ should be associated with at least one $WorkoutSession$.
  \item For {\em Personal Information}, making $emailAddress$ unique would be a good constraint as well.
\end{itemize}

\subsection*{\small d)}

Firstly, I would like to clarify the nature of the data we will store and its limits. What would we be storing precisely? Is weight in kilograms? How many decimal places should we support? Do client applications convert into imperial units if that's the user's preference? Getting answers to these questions will be vital to ensuring we capture all required attributes and translate them into appropriate SQL types that support these scenarios and optimise the underlying storage needs of our schema. It would also make sense to ask for clarifications about the relationships between datasets. Will we only send a single message when a user's goal has succeeded or failed? These answers will enable us to evaluate if any additional relations are needed and what their relationship will be with other relations in our schema, such as one-to-one or one-to many. \\
\newline
Secondly, I would want to gain additional insights into how this data will be queried. What are some typical user actions that may take place through the mobile app? What data will they want to see? This will clarify what indexes we want to apply across our schema. Each index will have a cost, impacting writes and increasing storage requirements, so we must use them sensibly. Identifying attributes we will join, filter, or aggregate would give us an initial set of fields to consider and allow us to determine the best type of index to apply for each scenario. For example, applying a clustered index on an attribute we use in a range query or a regular index for an attribute we filter by and select regularly. \\
\newline
Finally, I would ask questions about the required performance levels or expected scalability needs. How many reads will we need to support? Might we need to look at read replicas to help boost read speeds, vertically scale the database or potentially denormalise our design? How much data will we store, how many users do we need to support, and would it make sense to split data from users and data collected from sensors into separate database instances to partition high write traffic? Answers to these questions will fundamentally impact the design and implementation of our relational model. \\
\pagebreak

\section*{Q2}

For answering these questions I've assumed the presence of the $currentUserId$ variable which is of type $UserIdType$ that captures the user executing the query. I've also assumed that the $DateType$ type used in the schema has year, month and day variables that are natural numbers. Finally, I've assumed that $distance$ in the $DailyMovement$ relation is stored as miles and as such no conversion for the query is needed.

\subsection*{\small a)}

\begin{zed}
\{ d: DailyMovement | \\
    \t1 d.userId = currentUserId \land \\
    \t1 d.date.year = 2023 \land d.date.month = 2 \land \\
    \t1 d.distance \geq 1 \land d.pace \in \{ Walked, Ran \} \\
        \t2 @ \lblot day == d.date.day \rblot \}
\end{zed}

\subsection*{\small b)}
\begin{zed}
\{ g: Goal | \\
    \t1 g.userId = currentUserId \land \\
    \t1 g.startAt.month \leq 1 \land g.startAt.year \leq 2023 \\
    \t1 g.endOn.month \geq 1 \land g.endOn.year \geq 2023 \} \\
\end{zed}

\subsection*{\small c)}
\begin{zed}
A == Restrict\{userId = currentUserId \}(Goal) \\
B == Restrict\{achievedAt.year = 2023 \land status = Success \}(GoalOutcome) \\
C == Join(A, B) \\
D == Project\{achievedAt.year, achievedAt.month, goalId\}(C) \\
E == D\ GroupBy(Count(goalId)\ as\ metPerMonth) \\
F == Project\{year, metPerMonth\}(E) \\
G == F\ GroupBy(Max(metPerMonth)\ as\ maxGoalsMet) \\
H == Join(E, G) \\ 
I == Restrict\{maxGoalsMet = metPerMonth\}(H) \\
J == Project\{month\}(I) \\
\end{zed}

\subsection*{\small d)}
\begin{zed}
A == Restrict\{userId = currentUserId\}(DailyMovement) \\
B == Restrict\{date.year = 2023 \}(A) \\
C == Project\{date, calories\}(B) \\
D == C\ GroupBy(Sum(calories)\ as\ dailyTotal) \\ 
E == Project\{date.month, dailyTotal\}(D) \\ 
F == E\ GroupBy(Avg(dailyTotal)\ as\ dailyAverage) \\
\end{zed}

\pagebreak
\section*{Q3}
The SQL provided has been written to be compatible with MySQL v8.0 and would potentially need to be tweaked in circumstances for other relational database options.
\vspace{-0.25cm}
\subsection*{\small a)}
\vspace{-0.25cm}
\begin{lstlisting}[language=sql]
CREATE Table User (
  userId INT AUTO_INCREMENT PRIMARY KEY,
  emailAddress VARCHAR(100) UNIQUE NOT NULL,
  firstName VARCHAR(50) NOT NULL,
  lastName VARCHAR(50) NOT NULL,
  dateOfBirth DATE NOT NULL,
  gender ENUM('M', 'F', 'O') NOT NULL,
  height DECIMAL(5,2) NOT NULL,
  createdAt DATETIME NOT NULL
);

CREATE TABLE HealthReading (
  readingId INT AUTO_INCREMENT PRIMARY KEY,
  readAt DATETIME NOT NULL,
  userId INT NOT NULL,
  FOREIGN KEY (userId) REFERENCES User(userId)
);

CREATE TABLE HealthMetrics (
  metricsId INT AUTO_INCREMENT PRIMARY KEY,
  readingId INT NOT NULL,
  weight DECIMAL(5, 2),
  glucoseLevel DECIMAL(5, 2),
  systolicBloodPressure INT,
  diastolicBloodPressure INT,
  FOREIGN KEY (readingId) REFERENCES HealthReading(readingId)
);

CREATE TABLE Diet (
  dietId INT AUTO_INCREMENT PRIMARY KEY,
  forDate DATE NOT NULL,
  userId INT NOT NULL,
  FOREIGN KEY (userId) REFERENCES User(userId)
);

CREATE TABLE Meal (
  mealId INT AUTO_INCREMENT PRIMARY KEY,
  dietId INT NOT NULL,
  mealType ENUM('Breakfast', 'Lunch', 'Dinner') NOT NULL,
  mealSize ENUM('Light', 'Medium', 'Heavy') NOT NULL,
  mealTime TIME NOT NULL,
  FOREIGN KEY (dietId) REFERENCES Diet(dietId)
);

CREATE TABLE Food (
  foodId INT AUTO_INCREMENT PRIMARY KEY,
  dietId INT NOT NULL,
  foodName VARCHAR(255) NOT NULL,
  calories DECIMAL(5, 2) NOT NULL,
  quantity INT NOT NULL,
  consumedAt TIME NOT NULL,
  FOREIGN KEY (dietId) REFERENCES Diet(dietId)
);
\end{lstlisting}
\begin{lstlisting}[language=sql]
CREATE TABLE Drink (
  drinkId INT AUTO_INCREMENT PRIMARY KEY,
  dietId INT NOT NULL,
  drinkType ENUM('Water', 'Juice', 'Soda') NOT NULL,
  calories DECIMAL(5, 2) NOT NULL,
  glasses INT NOT NULL,
  consumedAt TIME NOT NULL,
  FOREIGN KEY (dietId) REFERENCES Diet(dietId)
);

CREATE TABLE Exercise (
  exerciseId INT AUTO_INCREMENT PRIMARY KEY,
  name VARCHAR(50) NOT NULL,
  description VARCHAR(100) NOT NULL
);

CREATE Table Workout (
  workoutId INT AUTO_INCREMENT PRIMARY KEY,
  startedAt DATETIME NOT NULL,
  userId INT NOT NULL,
  FOREIGN KEY (userId) REFERENCES User(userId)
);

CREATE TABLE WorkoutSession (
  sessionId INT AUTO_INCREMENT PRIMARY KEY,
  workoutId INT NOT NULL,
  exerciseId INT NOT NULL,
  weightLevel DECIMAL(5, 2) NOT NULL,
  numberOfSets INT NOT NULL,
  repetitionsPerSet INT NOT NULL,
  occuredAt TIME NOT NULL,
  FOREIGN KEY (workoutId) REFERENCES Workout(workoutId),
  FOREIGN KEY (exerciseId) REFERENCES Exercise(exerciseId)
);

CREATE TABLE DailyMovement (
  dailyMovementId INT AUTO_INCREMENT PRIMARY KEY,
  calories DECIMAL(6, 2) NOT NULL,
  distance DECIMAL(8, 2) NOT NULL,
  stairs INT NOT NULL,
  steps INT NOT NULL,
  pace ENUM('Walked', 'Ran', 'Jogged') NOT NULL,
  date DATE NOT NULL,
  userId INT NOT NULL,
  FOREIGN KEY (userId) REFERENCES User(userId)
);

CREATE TABLE HourlyMovement (
  hourlyMovementId INT AUTO_INCREMENT PRIMARY KEY,
  calories DECIMAL(6, 2) NOT NULL,
  distance DECIMAL(8, 2) NOT NULL,
  stairs INT NOT NULL,
  steps INT NOT NULL,
  pace ENUM('Walked', 'Ran', 'Jogged') NOT NULL,
  date DATE NOT NULL,
  time TIME NOT NULL,
  userId INT NOT NULL,
  FOREIGN KEY (userId) REFERENCES User(userId)
);
\end{lstlisting}
\begin{lstlisting}[language=sql]
CREATE TABLE Goal (
  goalId INT AUTO_INCREMENT PRIMARY KEY,
  userId INT NOT NULL,
  startOn DATE NOT NULL,
  endAt DATE NOT NULL,
  description VARCHAR(255) NOT NULL,
  objective ENUM('Weight Loss', 'Toning', 'Ran', 'Walked') NOT NULL,
  measurement ENUM('kg', 'km') NOT NULL,
  target DECIMAL(8, 2) NOT NULL,
  FOREIGN KEY (userId) REFERENCES User(userId)
);

CREATE TABLE GoalOutcome (
  outcomeId INT AUTO_INCREMENT PRIMARY KEY,
  goalId INT NOT NULL,
  status ENUM('Success', 'Failed') NOT NULL,
  achievedAt DATETIME NOT NULL,
  FOREIGN KEY (goalId) REFERENCES Goal(goalId)
);

CREATE TABLE GoalMessage (
  messageId INT AUTO_INCREMENT PRIMARY KEY,
  goalId INT NOT NULL,
  subject VARCHAR(50) NOT NULL,
  message TEXT NOT NULL,
  sentAt DATETIME NOT NULL,
  FOREIGN KEY (goalId) REFERENCES Goal(goalId)
);

CREATE TABLE GoalEvidence (
  evidenceId INT AUTO_INCREMENT PRIMARY KEY,
  goalId INT NOT NULL,
  dailyMovementId INT,
  hourlyMovementId INT,
  healthMetricsId INT,
  workoutSessionId INT,
  FOREIGN KEY (dailyMovementId) REFERENCES DailyMovement(dailyMovementId),
  FOREIGN KEY (hourlyMovementId) REFERENCES HourlyMovement(hourlyMovementId),
  FOREIGN KEY (healthMetricsId) REFERENCES HealthMetrics(metricsId),
  FOREIGN KEY (workoutSessionId) REFERENCES WorkoutSession(sessionId)
);
\end{lstlisting}


\pagebreak
\subsection*{\small b)}

\textbf{\small{C1}}
\begin{lstlisting}[language=sql]
ALTER TABLE Goal ADD CONSTRAINT Check_EndsAt CHECK (NOT endAt < startOn);
\end{lstlisting}
\textbf{\small{C2}}
\begin{lstlisting}[language=sql]
CREATE TRIGGER Check_DailyMovement
BEFORE INSERT ON DailyMovement
FOR EACH ROW
BEGIN
    IF (
        (NEW.calories, NEW.distance, NEW.stairs, NEW.steps) <> (
            SELECT COALESCE(SUM(calories), 0), COALESCE(SUM(distance), 0), 
              COALESCE(SUM(stairs), 0), COALESCE(SUM(steps), 0) 
            FROM HourlyMovement
            WHERE userId = NEW.userId and pace = NEW.pace and date = NEW.date
        )
    ) THEN
        SIGNAL SQLSTATE '45000' SET MESSAGE_TEXT = 'Daily/Hourly figures are not consistent';
    END IF;
END;
\end{lstlisting}
\textbf{\small{C3}}
\begin{lstlisting}[language=sql]
CREATE TRIGGER Check_ActiveGoals
BEFORE INSERT ON Goal
FOR EACH ROW
BEGIN
    IF (
      3 <= (
        SELECT COUNT(*) 
        FROM Goal 
        WHERE NEW.userId = userId AND NEW.endAt >= startOn AND NEW.startOn <= endAt
      ) 
    ) THEN
        SIGNAL SQLSTATE '45000' SET MESSAGE_TEXT = 'User cannot have more than 3 active goals';
    END IF;
END;
\end{lstlisting}
\textbf{\small{C4}}
\begin{lstlisting}[language=sql]
CREATE TRIGGER Check_GoalEvidence
BEFORE INSERT ON GoalOutcome
FOR EACH ROW
BEGIN
    IF (  
        0 = (
            SELECT COUNT(*) 
            FROM GoalEvidence 
            WHERE goalId = NEW.goalId AND (
              dailyMovementId IS NOT NULL OR 
              hourlyMovementId IS NOT NULL OR 
              workoutSessionId IS NOT NULL OR 
              healthMetricsId IS NOT NULL
            )
        )
    ) THEN
        SIGNAL SQLSTATE '45000' SET MESSAGE_TEXT = 'Goal outcome must have evidence';
    END IF;
END;
\end{lstlisting}

\pagebreak 
\subsection*{\small c)}
For these queries, there is a {\em @currentUserId} parameter assumed to be available to tailor the query to a given user, and I again assume distance is stored in miles so no converstion is needed. \\
\newline
\textbf{\small{1.}}
\begin{lstlisting}[language=sql]
SELECT DAY(date) as 'day'
FROM DailyMovement 
WHERE userId = @currentUserId AND 
  YEAR(date) = 2023 AND MONTH(date) = 2 AND 
  distance > 1 AND pace in ('Walked', 'Ran')
\end{lstlisting}
\textbf{\small{2.}}
\begin{lstlisting}[language=sql]
SELECT 
  goalId, 
  userId, 
  startOn, 
  endAt, 
  description, 
  objective, 
  measurement, 
  target
FROM Goal 
WHERE userId = @currentUserId AND 
  startOn < '2023-02-01' AND 
  endAt >= '2023-01-01'
\end{lstlisting}
\textbf{\small{3.}}
\begin{lstlisting}[language=sql]
SELECT MONTH(o.achievedAt) as 'month'
FROM Goal g
INNER JOIN GoalOutcome o ON g.goalId = o.goalId
WHERE YEAR(o.achievedAt) = 2023 AND 
  g.userId = @currentUserId AND 
  o.status = 'Success'
GROUP BY MONTH(o.achievedAt)
ORDER BY COUNT(g.goalId) DESC
LIMIT 1
\end{lstlisting}
\textbf{\small{4.}}
\begin{lstlisting}[language=sql]
SELECT 
  MONTH(date) as 'month', 
  AVG(totalDaily) as 'averageDailyCalories'
FROM (
  SELECT date, SUM(calories) 'totalDaily'
  FROM  DailyMovement 
  WHERE YEAR(date) = 2023 and userId = @currentUserId
  GROUP BY date
) T
GROUP BY MONTH(date)
\end{lstlisting}

\pagebreak

\section*{Q4}
\subsection*{\small a)}
The MongoDB design is outlined below. I have taken the existing 16 relations and cut them down to 6 collections. From a user's perspective, there are likely to be a relatively small number of writes daily, and therefore, catering to reads and queries spanning multiple days' worth of fitness data should be the priority. \\
\newline
In the design, I've strategically employed referencing and embedding patterns to strike a balance between reads and writes favouring reads. Multiple relations have been merged into a single collection, enhancing data access. Embedding data into a single document improves the speed of accessing daily diet, workout, and movement data, as it's all collated in the same document and avoids the need to perform joins or lookups. I've also used the referencing pattern to help maintain write performance where it makes sense. Storing user information in a separate $User$ collection with references kept in other collections to help reduce the document size when writing and updating specific sets of data. \\
\newline
I've leveraged various other design patterns to enhance read speeds and provide additional capabilities in the design. I've used the bucket design pattern on the $Diet$ and $Movement$ collections to store this data by date with all related information embedded, enabling more performant aggregation queries. The extended reference pattern is used for the storage of goal evidence, allowing us to store a subset of the data we're interested in in the $Goal$ collection, eliminating the need for lookups across multiple collections. Finally, I've added summary fields to the $Movement$ collection that can be sourced via the computed pattern, providing precalculated summaries to the application, reducing CPU workload and increasing application performance. \\
\newline
Using these patterns optimises reads at the expense of writes. Less referencing makes it harder to update pieces of daily data individually, and embedding means that regardless of whether you need all the data in the document, you will get it anyway. However, the design patterns here should help find a healthy balance, given the application's need to display and capture daily fitness data. \\
\newline
\hspace{-0.0cm}\textbf{User Collection}
\begin{lstlisting}[language=json]
{
  "_id": ObjectId("6670a844f76a17517b8b39ec"),
  "emailAddress": "EMAIL_ADDRESS",
  "firstName": "FIRST_NAME",
  "lastName": "LAST_NAME",
  "dateOfBirth": ISODate("1970-01-01T00:00:00.000Z"),
  "gender": "M|F|O",
  "height": 999.99,
  "createdAt": ISODate("1970-01-01T00:00:00.000Z"),
}
\end{lstlisting} 
\vspace{0.75cm}\textbf{Health Collection}
\begin{lstlisting}[language=json]
{
  "userId": ObjectId("6670a844f76a17517b8b39ec")
  "readAt": ISODate("1970-01-01T00:00:00.000Z")
  "metrics": [
    {
      "name": "WEIGHT|GLUCOSE|BLOOD_PRESSURE",
      "value": 9999999.9999,
      "unit": "kg|mg/dL|mmHg"
    }
  ]
}
\end{lstlisting} 
\pagebreak
\textbf{Workout Collection}
\begin{lstlisting}[language=json]
{
  "_id": ObjectId("6670ae6df76a17517b8b39fa")
  "userId": ObjectId("6670a844f76a17517b8b39ec")
  "startedAt": ISODate("1970-01-01T00:00:00Z"),
  "sessions": [
    {
      "exercise": {
        "name": "name",
        "description": "DESCRIPTION"
      },
      "weightLevel": 999.99,
      "numberOfSets": 1,
      "repititionsPerSet": 1,
      "occuredAt": ISODate("1970-01-01T00:00:00Z")
    }
  ]
}
\end{lstlisting} 
\vspace{0.75cm}\textbf{Goal Collection}
\begin{lstlisting}[language=json]
{
  "_id": ObjectId("6670b19df76a17517b8b3a03"),
  "userId": ObjectId("6670a844f76a17517b8b39ec"),
  "startOn": ISODate("1970-01-01:00:00:00Z"),
  "endAt": ISODate("1970-01-01:00:00:00Z"),
  "description": "DESCRIPTION",
  "objective": "WEIGHT_LOSS|TONING|RAN|WALKED",
  "measurement": "kg|km",
  "target": 999.99,
  "outcome": {
    "status": "SUCCESS|FAILED",
    "achievedAt": ISODate("1970-01-01:00:00:00Z")
  },
  "message": [
    {
      "subject": "SUBJECT",
      "message": "MESSAGE"
    }
  ],
  "evidence": [
    {
      "id": ObjectId("6670b084f76a17517b8b3a01"),
      "type": "MOVEMENT|HEALTH|WORKOUT"
    }
  ]
}
\end{lstlisting} 
\pagebreak
\textbf{Movement Collection}
\begin{lstlisting}[language=json]
{
  "_id": ObjectId("6670b084f76a17517b8b3a01"),
  "userId": ObjectId("6670a844f76a17517b8b39ec"),
  "forDate": ISODate("1970-01-01"),
  "hourly": [
  {
      "distance": 999.99,
      "calories": 999.99,
      "stairs": 0,
      "steps": 0,
      "pace": "WALKED|RAN|JOGGED",
      "time": ISODate("1970-01-01:00:00:00Z")
    }
  ]
  "daily": [
    {
      "distance": 999.99,
      "calories": 999.99,
      "stairs": 0,
      "steps": 0,
      "pace": "WALKED|RAN|JOGGED"
    }
  ],
  "total_calories": 999.99,
  "total_distance": 999.99
}
\end{lstlisting} 
\textbf{Diet Collection}
\begin{lstlisting}[language=json]
{
  "userId": ObjectId("6670a844f76a17517b8b39ec"),
  "forDate": ISODate("1970-01-01"),
  "meals": [
    {
      "name": "BREAKFAST|LUNCH|DINNER",
      "size": "LIGHT|MEDIUM|HEAVY",
      "time": ISODate("1970-01-01T00:00:00Z")
    }
  ],
  "food": [
    {
      "name": "FOOD_NAME",
      "calories": 999.99,
      "quantity": 1,
      "consumed": ISODate("1970-01-01T00:00:00Z")
    }
  ],
  "drinks": [
    {
      "name": "WATER|JUICE|SODA",
      "calories": 999.99,
      "glasses": 1,
      "consumed": ISODate("1970-01-01T00:00:00Z")
    }
  ]
}
\end{lstlisting} 
\pagebreak
\subsection*{\small b)}
I've provided the below queries as aggregation pipelines to get as close as possible to the result asked for in the question as I wouldn't be able to project or group with a basic MongoDB query. Like previous queries {\em @currentUserId} represents id of the current user and distance is stored in miles. \\
\newline
\hspace{0cm}\textbf{1.}
\begin{lstlisting}[language=json]
[
  {
    "$match": {
      "userId": ObjectId("@currentUserId"),
      "forDate": {
        "$gte": ISODate("2023-02-01T00:00:00Z"),
        "$lt": ISODate("2023-03-01T00:00:00Z")
      },
      "daily.distance": {
        "$gt": 1
      },
      "daily.pace": {
        "$in": ["WALKED", "RAN" ]
      }
    }
  },
  {
    "$project": {
      "_id": 0,
      "day": {
        "$dayOfMonth": "$forDate"
      }
    }
  }
]
\end{lstlisting} 
\hspace{0cm}\textbf{2.}
\begin{lstlisting}[language=json]
[
  {
    "$match": {
      "userId": ObjectId("@currentUserId"),
      "$or": [
        {
          "startOn": {
            "$lt": ISODate("2023-02-01T00:00:00Z")
          }
        },
        {
          "endAt": {
            "$gte": ISODate("2023-01-01T00:00:00Z")
          }
        }
      ]
    }
  }
]
\end{lstlisting}
\pagebreak
\hspace{0cm}\textbf{3.} 
\begin{lstlisting}[language=json]
[
  {
    "$match": {
      "userId": ObjectId("@currentUserId"),
      "outcome.status": "SUCCESS",
      "outcome.achievedAt": {
        "$gte": ISODate("2023-01-01T00:00:00.000Z"),
        "$lt": ISODate("2024-01-01T00:00:00.000Z")
      }
    }
  },
  {
    "$group" {
      "_id": {
        "$dateToString": {
          "format": "%m",
          "date": "$outcome.achievedAt"
        }
      },
      "count": {
        "$sum": 1
      }
    }
  },
  {
    "$sort": {
      "count": -1
    }
  },
  {
    "$limit": 1
  },
  {
    "$project": {
      "month": "$_id",
      "_id": 0
    }
  }
]
\end{lstlisting}
\pagebreak
\hspace{0cm}\textbf{4.}
\begin{lstlisting}[language=json]
[
  {
    "$match": {
      "userId": ObjectId("@currentUserId"),
      "forDate": {
        "$gte": ISODate("2023-01-01T00:00:00Z"),
        "$lt": ISODate("2024-01-01T00:00:00Z")
      }
    }
  },
  {
    "$unwind": "$daily"
  },
  {
    "$group": {
      "_id": "$forDate",
      "totalCalories": {
        "$sum": "$daily.calories"
      }
    }
  },
  {
    "$group": {
      "_id": {
        "$month": "$_id"
      },
      "averageDailyCalories": {
        "$avg": "$totalCalories"
      }
    }
  },
  {
    "$project": {
      "month": "$_id",
      "averageDailyCalories": 1,
      "_id": 0
    }
  }
]
\end{lstlisting}

\pagebreak

\subsection*{\small c)}

There are significant differences between MongoDB and a traditional relational database that impact the choice of which to leverage when building an application.\\
\newline
One of the most significant differentiating factors is data integrity and consistency. MongoDB offers flexible schemas that don't need to be declared in advance and validated when inserting documents, which is great for storing unstructured data or models that rapidly change. For relational databases, table columns have defined types to which rows inserted must conform. They also strengthen this capability by allowing users to declare assertions through check constraints and triggers that can add guards against malformed or inconsistent data, running checks when data is inserted, updated, or deleted to ensure integrity and consistency are not compromised. \\
\newline
Another significant factor is scaling requirements. Relational databases have a more limited set of levers when scaling out to handle more extensive data sets. They cannot scale horizontally as easily, and many have to fall back to vertical scaling, essentially throwing more hardware at the problem or through denormalisation or replicas, which come with their own sets of trade-offs. In contrast, MongoDB has built-in support for horizontal scaling through sharding strategies that can be applied at a collection level. Enabling complex and compute-heavy queries to be distributed across nodes in a shard cluster. \\
\newline
Both models support comprehensive querying capabilities but achieve them with very different implementations. Relational databases employ SQL, an industry standard that allows direct querying capabilities and is used to create the schema as well. Stored procedures can also be defined in SQL, providing a way to share SQL logic across applications that access the database directly. MongoDB has its own query language, but it's not a query language that can be used with any document-based database. It supports nearly everything you can find in SQL from a querying perspective, including joins, filters, and groups. However, this vendor lock-in will make transitioning from MongoDB to an alternative document database harder. \\
\newline
Both approaches have unique strengths and weaknesses. A relational database is hard to argue against if the application needs to ensure robust data consistency and integrity. However, MongoDB is an excellent choice if your application needs to scale horizontally and would benefit from a flexible schema definition. In our application scenario, you can argue for both. \\
\newline
The movement data tracked by sensors would be an excellent fit for MongoDB as it's likely to be the most read/write-intensive part of the application. We could leverage MongoDB's horizontal scaling through sharding to store and query large amounts of sensor data better, and should we need to be able to extract additional information from the sensors or want to change the frequency of collection, MongoDB's flexible schemas would help us accommodate that. \\
\newline
Meanwhile, manually entered user-facing data, such as goals, diet, and weight-lifting activities, would be a great fit for a relational database. We can apply constraints leveraging column types, checks and triggers to help stop invalid data and state from being introduced into the database as set out in the requirements. Solid ACID compliance will help us keep the database in a good state in unexpected error scenarios, and finally, with SQL, we can write complex queries optimised by the DBMS that efficiently grab the data we need to present to the user.\\
\end{document}