%
% tut.tex
%
% This file is part of fuzz2000
% Copyright (c) 1982--2006 J. M. Spivey
% All rights reserved
%
% Redistribution and use in source and binary forms, with or without
% modification, are permitted provided that the following conditions are met:
%
% 1. Redistributions of source code must retain the above copyright notice,
%    this list of conditions and the following disclaimer.
% 2. Redistributions in binary form must reproduce the above copyright notice,
%    this list of conditions and the following disclaimer in the documentation
%    and/or other materials provided with the distribution.
% 3. The name of the author may not be used to endorse or promote products
%    derived from this software without specific prior written permission.
%
% THIS SOFTWARE IS PROVIDED BY THE AUTHOR ``AS IS'' AND ANY EXPRESS OR
% IMPLIED WARRANTIES, INCLUDING, BUT NOT LIMITED TO, THE IMPLIED WARRANTIES
% OF MERCHANTABILITY AND FITNESS FOR A PARTICULAR PURPOSE ARE DISCLAIMED.
% IN NO EVENT SHALL THE AUTHOR BE LIABLE FOR ANY DIRECT, INDIRECT, INCIDENTAL,
% SPECIAL, EXEMPLARY, OR CONSEQUENTIAL DAMAGES (INCLUDING, BUT NOT LIMITED TO,
% PROCUREMENT OF SUBSTITUTE GOODS OR SERVICES; LOSS OF USE, DATA, OR PROFITS;
% OR BUSINESS INTERRUPTION) HOWEVER CAUSED AND ON ANY THEORY OF LIABILITY,
% WHETHER IN CONTRACT, STRICT LIABILITY, OR TORT (INCLUDING NEGLIGENCE OR
% OTHERWISE) ARISING IN ANY WAY OUT OF THE USE OF THIS SOFTWARE, EVEN IF
% ADVISED OF THE POSSIBILITY OF SUCH DAMAGE.
% 

\documentclass[12pt]{article}
\usepackage{fuzz}

\interdisplaylinepenalty=10000
\hyphenation{super-fluous data-bases}

\newcommand{\twinkle}{\par\bigskip
	\hbox to\linewidth{\hfil$*$\qquad$*$\qquad$*$\hfil}\bigskip}

\newcommand{\Array}{\mathrel{\bf array}}
\newcommand{\Of}{\mathrel{\bf of}}
\newcommand{\Procedure}{\mathrel{\bf procedure}}
\newcommand{\Var}{\mathrel{\bf var}}
\newcommand{\If}{\mathrel{\bf if}}
\newcommand{\Then}{\mathrel{\bf then}}
\newcommand{\Begin}{\mathrel{\bf begin}}
\newcommand{\End}{{\bf end}}
\newcommand{\While}{\mathrel{\bf while}}
\newcommand{\Do}{\mathrel{\bf do}}

\begin{document}

\title{An Introduction to Z and Formal Specifications}
\author{J. M. Spivey\\
	Oxford University Computing Laboratory\\
	Programming Research Group\\
	11, Keble Road, Oxford, OX1 3QD}
\date{June 1988}
\maketitle

\section*{Abstract}

This article is an introduction to the description of information
systems using formal, mathematical specifications written in the Z
notation, and to the refinement of these specifications into
rigorously-checked designs.

The first part introduces the idea of a formal specification using a
simple example: that of a ``birthday book'' in which people's
birthdays can be recorded, and which is able to issue reminders on the
appropriate day.  The behaviour of this system for correct input is
specified first, then the schema calculus is used to strengthen the
specification into one requiring error reports for incorrect input.

The second part of the article introduces the idea of data refinement as
the primary means of constructing designs which achieve a formal
specification. Refinement is presented through the medium of two
examples; the first is a direct implementation of the birthday
book from part one, and the second is a simple checkpoint facility,
which allows the current state of a database to be saved and later
restored. A Pascal-like programming language is used to show
the code for some of the operations in the examples.

\clearpage
\section{What is a formal specification?}

Formal specifications use mathematical notation to describe in a
precise way the properties which an information system must have,
without unduly constraining the way in which these properties are
achieved. They describe {\em what\/} the system must do without saying
{\em how\/} it is to be done.  This {\em abstraction\/} makes formal
specifications useful in the process of developing a computer system,
because they allow questions about what the system does to be answered
confidently, without the need to disentangle the information from a
mass of detailed program code, or to speculate about the meaning of
phrases in an imprecisely-worded prose description.

A formal specification can serve as a single, reliable reference point
for those who investigate the customer's needs, those who implement programs
to satisfy those needs, those who test the results, and those who write
instruction manuals for the system.
Because it is independent of the program code, a formal specification of
a system can be completed early in its development. Although it
might need to be changed as the design team gains in understanding and
the perceived needs of the customer evolve, it can be a valuable means
of promoting a common understanding among all those concerned with the
system.

Existing data modelling techniques -- for example,
relational databases~\cite{Date} -- provide some of the
abstraction we need, in that they free us from describing the
exact layout of data in the memory of a computer. But they
are limited to fairly simple models, and they are oriented
towards direct implementation: considerations of efficiency
rather than clarity often dictate the structure of the
model. Also, whilst data modelling techniques can help to
clarify the description of a system, they provide little
support for reasoning about its behaviour.

One way in which mathematical notation can help to achieve these
goals is through the use of {\em mathematical data types\/}
to model the data in a system. These data types
are not oriented towards computer representation, but they
obey a rich collection of mathematical laws which make it possible
to reason effectively about the way a specified system will
behave. We use the notation of {\em predicate logic\/} to
describe abstractly the effect of each operation of our
system, again in a way that enables us to reason about its
behaviour.
These two ideas also form important ingredients of other formal methods
such as VDM~\cite{Jones80,Jones86}.

The other main ingredient in Z is a way of decomposing
a specification into small pieces called {\em schemas}. By
splitting the specification into schemas, we can present
it piece by piece. Each piece can be linked with a
commentary which explains informally the significance of the
formal mathematics.
In Z, schemas are used to describe both
static and dynamic aspects of a system. The static aspects
include 
\begin{itemize}
\item	the states it can occupy.
\item	the invariant relationships that are maintained
	as the system moves from state to state.
\end{itemize}
The dynamic aspects include
\begin{itemize}
\item	the operations that are possible.
\item	the relationship between their inputs and outputs.
\item	the changes of state that happen.
\end{itemize}
Later, we shall see how the schema language allows different
facets of a system to be described separately, then related
and combined. For example, the operation of a system when it
receives valid input may be described first, then the
description may be extended to show how errors in the input
are handled. Or the evolution of a single process in a
complete system may be described in isolation, then related
to the evolution of the system as a whole.

We shall also see how schemas can be used to describe a
transformation from one view of a system to another, and so
explain why an abstract specification is correctly
implemented by another containing more details of a concrete
design. By constructing a sequence of specifications, each
containing more details than the last, we can eventually
arrive at a program with confidence that it satisfies the
specification.

\section{The birthday book}\label{bb}

The best way to see how these ideas work out is to look at a
small example. For a first example, it is important to
choose something simple, and I have chosen a system so
simple that it is usually implemented with a notebook and
pencil rather than a computer. It is a system which records
people's birthdays, and is able to issue a reminder when the
day comes round.
The first thing to describe is the {\em state space\/} of
the system, and we do this with a schema:
%% \begin{zed} [NAME, DATE] \end{zed}
\begin{schema}{BirthdayBook}
    known: \power NAME \\
    birthday: NAME \pfun DATE
\where
    known=\dom birthday
\end{schema}
Like most schemas, this consists of a part above the central
dividing line, in which some variables are declared, and a
part below the line which gives a relationship between the
values of the variables. In this case we are describing the
state space of a system, and the two variables represent
important {\em observations\/} which we can make of the
state:
\begin{itemize}
\item $known$ is the set of names with birthdays recorded.
\item $birthday$ is a function which, when applied to certain
names, gives the birthdays associated with them.
\end{itemize}
The part of the schema below the line gives a relationship
which is true in every state of the system and is maintained
by every operation on it: in this case, it says that the set
$known$ is the same as the domain of the function $birthday$
-- the set of names to which it can be validly applied.
This relationship is an {\em invariant\/} of the system.

In this example, the invariant allows the the value of the variable
$known$ to be derived from the value of $birthday$: $known$ is a
{\em derived\/} component of the state, and it would be possible to
specify the system without mentioning $known$ at all.  However,
giving names to important concepts helps to make specifications more
readable; because we are describing an abstract view of the state space
of the birthday book, we can do this without making a
commitment to represent $known$ explicitly in an implementation.

One possible state of the system is the following:
\[
	known = \{\,{\rm John, Mike, Susan}\,\}
\also
	birthday = \{\,\vtop{\halign{\strut#\hfil&${}\mapsto{}$#\hfil\cr
			John&  25--Mar,\cr
			Mike&  20--Dec,\cr
			Susan& 20--Dec\,\}.\cr}}
\]
Here there are three names known to the system, and the
birthday function associates a date with each of them.

Notice that in this description of the state space of the
system, we have not been forced to place a limit on the
number of birthdays recorded in the birthday book, 
nor  to say that the
entries will be stored in a particular order. We have also
avoided making a premature decision about the format of
names and dates. On the other hand, we have concisely
captured the information that each person can have only one
birthday, because the variable $birthday$ is a function, and
that two people can share the same birthday as in our
example.

So much for the state space; we can now start on some {\em
operations\/} on the system. The first of these is to add a
new birthday, and we describe it with a schema:
\begin{schema}{AddBirthday}
     \Delta BirthdayBook \\
     name?: NAME \\
     date?: DATE
\where
     name? \notin known
\also
     birthday' = birthday \cup \{name? \mapsto date?\}
\end{schema}
The declaration $\Delta BirthdayBook$ alerts us to the fact
that the schema is describing a {\em state change}: it
introduces four variables $known$, $birthday$, $known'$ and
$birthday'$. The first two are observations of the state before the
change, and the last two are observations of the state after the
change. Each pair of variables is implicitly constrained to satisfy
the invariant, so it must hold both before and after the operation.
Next come the declarations of the two inputs to the operation.
By convention, the names of inputs end in a question mark.

The part of the schema below the line first of all gives a
{\em pre-condition\/} for the success of the operation: the
name to be added must not already be one of those known to
the system. This is reasonable, since each person can only
have one birthday. What happens if the pre-condition is
not satisfied is not specified here: we shall see later how
to extend the specification to say that an error message is
to be produced. If the pre-condition is satisfied, however, the
second line says that the birthday function is extended to
map the new name to the given date.

We expect that the set of names known to the system will be
augmented with the new name:
\[
	known' = known \cup \{name?\}.
\]
In fact we can {\em prove\/} this from the specification of
$AddBirthday$, using the invariants on the state before and
after the operation:      
\begin{argue}
	known' \\
\t1	= \dom birthday' & 			invariant after \\
\t1    	= \dom (birthday \cup \{name? \mapsto date?\}) &
						spec.\ of $AddBirthday$ \\
\t1	= \dom birthday \cup \dom\,\{name? \mapsto date?\} &
						fact about $\dom$ \\
\t1	= \dom birthday \cup \{name?\} &	fact about $\dom$ \\
\t1	= known \cup \{name?\}. & 		invariant before
\end{argue}
Stating and proving properties like this one is a good way
of making sure the specification is accurate; reasoning from
the specification allows us to explore the behaviour
of the system without going to the trouble and expense of
implementing it.
The two facts about $\dom$ used in this proof are examples
of the laws obeyed by mathematical data types:
\[
	\dom (f \cup g) = (\dom f) \cup (\dom g) \\
\also
	\dom \{a \mapsto b\} = \{a\}.
\]
The standard mathematical data types of Z obey many laws like these.

Another operation might be to find the
birthday of a person known to the system. Again we describe
the operation with a schema:
\begin{schema}{FindBirthday}
	\Xi BirthdayBook \\
	name?: NAME \\
	date!: DATE 
\where
	name? \in known
\also
	date! = birthday(name?)
\end{schema}
Two new notations are illustrated by this schema.  One is the
declaration $\Xi BirthdayBook$, which indicates an operation in which
the state does not change: the values $known'$ and $birthday'$ of the
observations after the operation are equal to their values $known$ and
$birthday$ beforehand. Including $\Xi BirthdayBook$ above the line is
the same as including $\Delta BirthdayBook$ above the line and
the two equations:
\[
	known' = known
\also
	birthday' = birthday
\]
below it. The other notation is the use of a name ending in
an exclamation mark for an output: the
$FindBirthday$ operation takes a name as input and yields
the corresponding birthday as output.
The pre-condition for success of the operation is that
$name?$ is one of the names known to the system; if this is
so, the output $date!$ is the value of the birthday function
at argument $name?$.

The most useful operation on the system is the one to find
which people have birthdays on a given date.  The operation has one
input $today?$, and one output, $cards!$, which is a {\em set\/} of names:
there may be zero, one, or more people with birthdays on a
particular day, to whom birthday cards should be sent.
\begin{schema}{Remind}
	\Xi BirthdayBook \\
	today?: DATE \\
	cards!: \power NAME
\where
	cards! = \{\,n: known | birthday(n) = today?\,\}
\end{schema}
Again the $\Xi$ convention is used to indicate that the
state does not change. This time there is no pre-condition.
The output $cards!$ is specified to be equal to the set of
all values $n$ drawn from the set $known$ such that the
value of the birthday function at $n$ is $today?$. In general,
$y$ is a member of the set $\{\,x:S|\ldots x\ldots\,\}$
exactly if $y$ is a member of $S$ and the condition $\ldots
y \ldots$, obtained by replacing $x$ with $y$, is satisfied:
\[
	y \in \{\,x: S | \ldots x \ldots \,\}
		\iff y \in S \land (\ldots y \ldots).
\]
So, in our case,
\[
	m \in \{\,n: known | birthday(n) = today?\,\} \\
\t1 		\iff m \in known \land birthday(m) = today?.
\]
A name $m$ is in the output set $cards!$ exactly if it is
known to the system and the birthday recorded for it is
$today?$.

To finish the specification, we must say what state the system is in
when it is first started. This is the {\em initial state\/} of the
system, and it also is specified by a schema:
\begin{schema}{InitBirthdayBook}
	BirthdayBook
\where
	known = \emptyset
\end{schema}
This schema describes a birthday book in which the set $known$ is
empty: in consequence, the function $birthday$ is empty too.

What have we achieved in this specification? We have described in the
same mathematical framework both the state space of our birthday-book
system and the operations which can be performed on it.  The data
objects which appear in the system were described in terms of
mathematical data types such as sets and functions. The description of
the state space included an invariant relationship between the parts of
the state -- information which would not be part of a program
implementing the system, but which is vital to understanding it.

The effects of the operations are described in terms of the
relationship which must hold between the input and the
output, rather than by giving a recipe to be followed. This
is particularly striking in the case of the $Remind$
operation, where we simply documented the conditions under
which a name should appear in the output. An implementation
would probably have to examine the known names one at a
time, printing the ones with today's date as it found them,
but this complexity has been avoided in the specification. The
implementor is free to use this technique, or any other one,
as he or she chooses.

Mathematical specifications have the three virtues of being concise,
precise and unambiguous.  They are {\em concise\/} because
mathematical notation is capable of expressing complex facts about
information systems in a short space.  Practical experience shows that
a mathematical specification of a system is often much shorter than an
equivalent informal specification.  Hayes~\cite{Hayes} reports that a
formal specification for a module in the CICS system is comparable in
length with the less informative English-language manual entry for the
same module.  Mathematical specifications are {\em precise\/} because
they allow requirements to be documented accurately. The desired
function of a system is described in a way that does not unduly
constrain either the data structures used to represent the information
in the system, or the algorithms used to compute with it.  Finally,
mathematical specifications are {\em unambiguous}: differences of
interpretation can be avoided when specifications are expressed in a
standardized language with a well-understood meaning.

\paragraph{Exercise.}
We have seen how schemas can be used to describe the
state space and operations of a system. Now try to write a Z
specification for the following system:
a teacher wants to keep a register of students in her class,
and to record which of them have completed their homework.
Specify:
\begin{enumerate}
\item	The state space for a register. [Hint: use two sets
	of students
%%unchecked
	\begin{schema}{Register}
		enrolled: \power STUDENT \\
		completed: \power STUDENT
	\where
		\ldots
	\end{schema}
	Think carefully about the invariant].
\item	An operation to enroll a new student.
\item	An operation to record that a student (already
	enrolled in the class) has finished the homework.
\item	An operation to enquire whether a student (who must
	be enrolled) has finished the homework. (Answer in
	the set $\{Yes, No\}$).
\end{enumerate}

\section{Strengthening the specification}

A correct implementation of our specification will faithfully record
birthdays and display them, so long as there are no mistakes in the
input. But the specification has a serious flaw: as soon as the user
tries to add a birthday for someone already known to the system, or
tries to find the birthday of someone not known, it says nothing about
what happens next. The action of the system may be perfectly
reasonable: it may simply ignore the incorrect input. On the other
hand, the system may break down: it may start to display rubbish, or
perhaps worst of all, it may appear to operate normally for several
months, until one day it simply forgets the birthday of a rich and
elderly relation.

Does this mean that we should scrap the specification and begin a new
one? That would be a shame, because the specification we have describes
the behaviour for correct input clearly and concisely, and modifying it
to describe the handling of incorrect input could only make it obscure.
Luckily, there is a better solution: we can describe, separately from
the first specification, the errors which might be detected and the
desired responses to them, then use the operations of the Z {\em schema
calculus\/} to combine the two descriptions into a stronger
specification.

We shall add an extra output $result!$ to each operation on the system.
When an operation is successful, this output will take the value $ok$,
but it may take other values when an error is detected. We first
describe an operation $Success$ which just produces the result $ok$:
%% \begin{zed} REPORT ::= ok | already\_known | not\_known \end{zed}
\begin{schema}{Success}
	result!: REPORT
\where
	result! = ok
\end{schema}
The conjunction operator $\land$ of the schema calculus allows us to
combine this description with our previous description of $AddBirthday$:
\[
	AddBirthday \land Success.
\]
This describes an operation which, for correct input, both acts as
described by $AddBirthday$ and produces the result $ok$.

For each error that might be detected in the input, we specify an
operation which produces an appropriate report when the error has
occurred. Here is an operation which produces the report
$already\_known$ when its input $name?$ is already a member of $known$:
\begin{schema}{AlreadyKnown}
	\Xi BirthdayBook \\
	name?: NAME \\
	result!: REPORT
\where
	name? \in known
\also
	result! = already\_known
\end{schema}
If the error occurs, this schema specifies that the state of the system
should not change.

We can combine this description with the previous one to give a
specification for a robust version of $AddBirthday$:
\begin{zed}
	RAddBirthday \defs (AddBirthday \land Success) \lor AlreadyKnown.
\end{zed}
This definition written with the sign $\defs$ introduces a new
schema called $RAddBirthday$, obtained by combining the three schemas shown
on the right-hand side.
The operation $RAddBirthday$ must terminate whatever its input.
If the input $name?$ is already known, the state of the system does not
change, and the result $already\_known$ is returned; otherwise,
the new birthday is added to the
database as described by $AddBirthday$, and the result $ok$ is returned.

We have specified the various requirements for this operation separately,
and then combined them into a single specification of the whole
behaviour of the operation. This does not mean that each requirement
must be implemented separately, and the implementations combined
somehow.
In fact, an implementation might search for a place to store the new
birthday, and at the same time check that the name is not already known;
the code for normal operation and error handling might be thoroughly
mingled.
This is an example of the abstraction which is possible when we use a
specification language free from the constraints necessary
in a programming language. The operators $\land$ and $\lor$
cannot (in general) be implemented efficiently as ways of combining
programs, but this should not stop us from using them to combine
specifications if that is a convenient thing to do.

The operation $RAddBirthday$ could be specified directly by writing
a single schema which combines the predicate parts of the three
constituents $AddBirthday$, $Success$ and $AlreadyKnown$:
%%unchecked
\begin{schema}{RAddBirthday}
	\Delta BirthdayBook \\
	name?: NAME \\
	date?: DATE \\
	result!: REPORT
\where
	(name? \notin known \land \\
\t1		birthday' = birthday \cup \{name? \mapsto date?\} \land \\
\t1		result! = ok) \lor \\
	(name? \in known \land \\
\t1		birthday' = birthday \land \\
\t1		result! = already\_known)
\end{schema}
As you can see, the effect of the schema $\lor$ operator is to make a schema
in which the predicate part is the result of joining the predicate parts of
its two arguments with the logical connective $\lor$. Similarly, the effect
of the schema $\land$ operator is to take the conjunction of the two
predicate parts.
Any common variables of the two schemas are merged: in this example, the
input $name?$, the output $result!$, and the four observations of the
state before and after the operation are shared by the two arguments of
$\lor$.
In order to write $RAddBirthday$ as a single schema, it has been necessary
to write out explicitly something which was implicitly part of the declaration
$\Xi BirthdayBook$, namely that the state doesn't change.

A robust version of the $FindBirthday$ operation must be able to report
if the input name is not known:
\begin{schema}{NotKnown}
	\Xi BirthdayBook \\
	name?: NAME \\
	result!: REPORT
\where
	name? \notin known
\also
	result! = not\_known
\end{schema}
The robust operation either behaves as described by $FindBirthday$ and
reports success, or reports that the name was not known:
\begin{zed}
	RFindBirthday \defs (FindBirthday \land Success) \lor NotKnown.
\end{zed}
The $Remind$ operation can be called at any time: it never results in
an error, so the robust version need only add the reporting of success:
\begin{zed}
	RRemind \defs Remind \land Success.
\end{zed}

The separation of normal operation from error-handling which we
have seen here is the simplest but also the most common kind of
modularization possible with the schema calculus.
More complex modularizations include {\em promotion} or
{\em framing}~\cite{UNIX}, where operations
on a single entity -- for example, a file -- are made into
operations on a named entity in a larger system -- for example, a
named file in a directory.
The operations of reading and writing a file might be described by
schemas. Separately, another schema might describe the way a file can
be accessed in a directory under its name. Putting these two parts
together would then result in a specification of operations for
reading and writing named files.

Other modularizations are possible: for example, the
specification of a system with access restrictions might separate the
description of who may call an operation from the description of what
the operation actually does.
There are also facilities for generic definitions in Z which allow, for
example, the notion of resource management to be specified in general,
then applied to various aspects of a complex system~\cite{CAVIAR}.

\paragraph{Exercise.}
Specify a robust version of the class register system.

\section{From specifications to designs}

We have seen how the Z notation can be used to specify software modules,
and how the schema calculus allows us to put together the specification
of a module from pieces which describe various facets of its function.
Now we turn our attention to the techniques used in Z to document the design
of a program which implements the specification.

The central idea is to describe the concrete data structures which the
program will use to represent the abstract data in the specification,
and to derive descriptions of the operations in terms of the concrete
data structures. We call this process {\em data refinement}.  Often, a
data refinement will allow some of the control structure of the
program to be made explicit, and this is achieved by one or more steps
of {\em operation refinement\/} or {\em algorithm development}.

For simple systems, it is possible to go from the abstract specification
to the final program in one step, a method sometimes called {\em direct
refinement}. In more complex systems, however, there are too many design
decisions for them all to be recorded clearly in a single refinement step,
and the technique of {\em deferred refinement\/} is appropriate.
Instead of a finished program, the first refinement step results in a new
specification, and this is then subjected to further steps of refinement until
a program is at last reached. The result is a sequence of design documents,
each describing a small collection of related design decisions. As the
details of the data structures are filled in step by step, so more of the
control structure can be filled in, leaving certain sub-tasks to be
implemented in subsequent refinement steps. These sub-tasks can be made
into subroutines in the final program, so the step-wise structure of the
development leads to a modular structure in the program.

Program developments are often documented by giving an idealized
account of the path from specification to program. In these accounts,
the ideas all appear miraculously at the right time, one after
another. There are no mistakes, no false starts, no decisions taken
which are later revised. Of course, real program developments don't
happen like that, and the earlier stages of a development are often
revised many times as later stages cast new light on the system. In
any case, specifications are seldom written without at least a rough
idea of how they might be implemented, and it is very rare to find
that something similar hasn't been implemented before.

This doesn't mean that the idealized accounts are worthless, however.
They are often the best way of presenting the decisions which have been made
and the relationships between them, and such an account can be a valuable
piece of documentation, even if it is economical with the true history
of the development.

The rest of this article concentrates on data refinement in Z,
although the results of the operation refinement which might follow it
are shown.  Two examples of data refinement are presented.  The first
shows direct refinement: the birthday book we specified in
section~\ref{bb} is implemented using a pair of arrays. In the second
example, deferred refinement is used to show the implementation of a
simple checkpoint-restart mechanism.  The implementation uses two
sub-modules for which specifications in Z are derived as part of the
refinement step. This demonstrates the way in which mathematics can
help us to explore design decisions at a high level of abstraction.

\section{Implementing the birthday book}

The specification of the birthday book worked with abstract data structures
chosen for their expressive clarity rather than their ability to be directly
represented in a computer. In the implementation, the data structures must
be chosen with an opposite set of criteria, but they can still be modelled
with mathematical data types and documented with schemas.

In our implementation, we choose to represent the birthday book with two
arrays, which might be declared by
\[
	names:\;\Array [1\upto{}] \Of NAME; \\
	dates:\;\Array [1\upto{}] \Of DATE;
\]
I have made these arrays ``infinite'' for the sake of simplicity. In
a real system development, we would use the schema calculus to specify
a limit on the number of entries, with appropriate error reports if
the limit is exceeded. Finite arrays could then be used in a more
realistic implementation; but for now, this would just be a distraction,
so let us pretend that potentially infinite arrays are part of our
programming language. We shall, in any case, only use a finite part of
them at any time.

These arrays can be modelled mathematically by functions from the set
$\nat_1$ of strictly positive integers to $NAME$ or $DATE$:
\[
	names: \nat_1 \fun NAME \\
	dates: \nat_1 \fun DATE.
\]
The element $names[i]$ of the array is simply the value $names(i)$ of
the function, and the assignment
$names[i] := v$
is exactly described by the specification
\[
	names' = names \oplus \{i\mapsto v\}.
\]
The right-hand side of this equation is a function which takes the
same value as $names$ everywhere except at the argument~$i$, where it
takes the value~$v$.

We describe the state space of the program as a schema. There is
another variable $hwm$ (for ``high water mark''); it shows how much of
the arrays is in use.
\begin{schema}{BirthdayBook1}
	names: \nat_1 \fun NAME \\
	dates: \nat_1 \fun DATE \\
	hwm: \nat
\where
	\forall i, j: 1 \upto hwm @ \\
\t1		i \neq j \implies names(i) \neq names(j)
\end{schema}
The predicate part of this schema says that there are no repetitions
among the elements $names(1)$, \dots, $names(hwm)$.

The idea of this representation is that each name is linked with the
date in the corresponding element of the array $dates$. We can
document this by defining another schema $Abs$ that defines the {\em
abstraction relation\/} between the abstract state space
$BirthdayBook$ and the concrete state space $BirthdayBook1$:
\begin{schema}{Abs}
	BirthdayBook \\
	BirthdayBook1
\where
	known = \{\,i: 1 \upto hwm @ names(i)\,\}
\also
	\forall i: 1 \upto hwm @ \\
\t1		birthday(names(i)) = dates(i)
\end{schema}

%\pagebreak
\noindent This schema relates two points of view on the state of the system.
The observations involved are both those of the abstract state --
$known$ and $birthday$ -- and those of the concrete state -- $names$,
$dates$ and $hwm$.  The first predicate says that the set $known$
consists of just those names which occur somewhere among $names(1)$,
\dots,~$names(hwm)$.
The set $\{\,y: S @ \ldots y \ldots\,\}$ contains those values
taken by the expression $\ldots y \ldots$ as $y$ takes values in the
set $S$, so $known$ contains a name $n$ exactly if $n = names(i)$ for
some value of $i$ such that $1 \leq i \leq hwm$. We can write this in
symbols with an existential quantifier:
\[
	n \in known \iff (\exists i: 1 \upto hwm @ n = names(i)).
\]
The second predicate says that the birthday for $names(i)$ is the
corresponding element $dates(i)$ of the array~$dates$.

Several concrete states may represent the same abstract state: in the
example, the order of the names and dates in the arrays does not
matter, so long as names and dates correspond properly. The order is
not used in determining which abstract state is represented by a
concrete state, so two states which have the same names and dates in
different orders will represent the same abstract state.  This is quite
usual in data refinement, because efficient representations of data
often cannot avoid including superfluous information.

On the other hand, each concrete state represents only one abstract
state.  This too is the usual situation, because we don't expect to
find superfluous information in the abstract state. It does sometimes
happen that one concrete state represents several abstract states, but
this is often a sign of a badly-written specification that has a bias
towards a particular implementation.

Having explained what the concrete state space is, and how concrete states
are related to abstract states, we can begin to implement the operations
of the specification.
To add a new name, we increase $hwm$ by one, and fill in the name and date
in the arrays:
\begin{schema}{AddBirthday1}
	\Delta BirthdayBook1 \\
	name?: NAME \\
	date?: DATE
\where
	\forall i: 1 \upto hwm @ name? \neq names(i)
\also
	hwm' = hwm + 1 \\
	names' = names \oplus \{ hwm' \mapsto name? \} \\
	dates' = dates \oplus \{ hwm' \mapsto date? \}
\end{schema}
This schema describes an operation which has the same inputs and
outputs as $AddBirthday$, but operates on the concrete instead of the abstract
state. It is a correct implementation of $AddBirthday$, because of the
following two facts:
\begin{enumerate}
\item	Whenever $AddBirthday$ is legal in some abstract state,
	the implementation $AddBirthday1$ is legal in any corresponding
	concrete state.
\item	The final state which results from $AddBirthday1$ represents
	an abstract state which $AddBirthday$ could produce.
\end{enumerate}
Let us look at the reasons why these two facts are true.
The operation $AddBirthday$ is legal exactly if its pre-condition
$name? \notin known$ is satisfied.  If this is so, the predicate
\[
	known = \{\,i: 1 \upto hwm @ names(i)\,\}
\]
from $Abs$ tells us that $name?$ is not one of the elements $names(i)$:
\[
	\forall i: 1 \upto hwm @ name? \neq names(i).
\]
This is the pre-condition of $AddBirthday1$.

To prove the second fact, we need to think about the concrete states before
and after an execution of $AddBirthday1$, and the abstract states they
represent according to $Abs$. 
The two concrete states are related by $AddBirthday1$,
and we must show that the two abstract states are related as
prescribed by $AddBirthday$:
\[
	birthday' = birthday \cup \{name? \mapsto date?\}.
\]
The domains of these two functions are the same, because
\begin{argue}
	\dom birthday' \\
\t1	= known' &			invariant after \\
\t1	= \{\,i: 1 \upto hwm' @ names'(i)\,\} & 
					from $Abs'$ \\
\t1	= \{\,i: 1 \upto hwm @ names'(i)\,\} \cup \{names'(hwm')\} \\
		& since $hwm' = hwm + 1$ \\
\t1	= \{\,i: 1 \upto hwm @ names(i)\,\} \cup \{name?\} \\
		& since $names' = names \oplus \{hwm' \mapsto name?\}$ \\
\t1	= known \cup \{name?\} &	from $Abs$ \\
\t1	= \dom birthday \cup \{name?\}. & invariant before
\end{argue}
There is no change in the part of the arrays which was
in use before the operation, so for all $i$ in the range $1 \upto hwm$,
\[
	names'(i) = names(i) \land dates'(i) = dates(i).
\]
For any $i$ in this range,
\begin{argue}
	birthday'(names'(i)) \\
\t1	= dates'(i) & 			from $Abs'$ \\
\t1	= dates(i) & 			$dates$ unchanged \\
\t1	= birthday(names(i)). &		from $Abs$
\end{argue}
For the new name, stored at index $hwm' = hwm+1$,
\begin{argue}
	birthday'(names'(hwm')) \\
\t1	= dates'(hwm') &		from $Abs'$ \\
\t1	= date?. &			spec.\ of $AddBirthday1$
\end{argue}
So the two functions $birthday'$ and $birthday \cup \{name? \mapsto date?\}$
are equal, and the abstract states before and after the operation are
guaranteed to be related as described by $AddBirthday$.

The description of the concrete operation uses only notation which has a
direct counterpart in our programming language, so we can translate it
directly into a subroutine to perform the operation:
\[
	\Procedure AddBirthday(name: NAME; date: DATE); \\
	\Begin \\
\t1		hwm := hwm + 1; \\
\t1		names[hwm] := name; \\
\t1		dates[hwm] := date \\
	\End;
\]

%\pagebreak
The second operation, $FindBirthday$, is implemented by the following
operation, again described in terms of the concrete state:
\begin{schema}{FindBirthday1}
	\Xi BirthdayBook1 \\
	name?: NAME \\
	date!: DATE
\where
	\exists i: 1 \upto hwm @ \\
\t1		name? = names(i) \land date! = dates(i)
\end{schema}
The predicate says that there is an index $i$ at which the $names$
array contains the input $name?$, and the output $date!$ is the
corresponding element of the array $dates$.  For this to be possible,
$name?$ must in fact appear somewhere in the array $names$: this is
the pre-condition of the operation. 

Since neither the abstract nor the concrete operation changes the
state, there is no need to check that the final concrete state is
acceptable, but we need to check that the pre-condition of
$FindBirthday1$ is sufficiently liberal, and that the output $date!$
is correct.  The pre-conditions of the abstract and concrete operations
are in fact the same: that the input $name?$ is known. The output is
correct because for some~$i$, $name? = names(i)$ and $date! =
dates(i)$, so
\begin{argue}
	date! \\
\t1	= dates(i) &			spec.\ of $FindBirthday1$ \\
\t1	= birthday(names(i)) &		from $Abs$ \\
\t1	= birthday(name?). &		spec.\ of $FindBirthday1$
\end{argue}
The existential quantifier in the description of $FindBirthday1$ leads to
a loop in the program code, searching for a suitable value of~$i$:
\[
	\Procedure FindBirthday(name: NAME; \Var date: DATE); \\
\t1		\Var i: INTEGER; \\
	\Begin \\
\t1		i := 1; \\
\t1		\While names[i] \neq name \Do i := i+1; \\
\t1		date := dates[i] \\
	\End;
\]

%\pagebreak
The operation $Remind$ poses a new problem, because its output $cards$
is a {\em set\/} of names, and cannot be directly
represented in the programming language. We can deal with it by introducing
a new abstraction relation, showing how it can be represented by an array
and an integer:
\begin{schema}{AbsCards}
	cards: \power NAME \\
	cardlist: \nat_1 \fun NAME \\
	ncards: \nat
\where
	cards = \{\,i: 1 \upto ncards @ cardlist(i)\,\}
\end{schema}
The concrete operation can now be described: it produces as outputs
$cardlist$ and $ncards$:
\begin{schema}{Remind1}
	\Xi BirthdayBook1 \\
	today?: DATE \\
	cardlist!: \nat_1 \fun NAME \\
	ncards!: \nat
\where
	\{\,i: 1 \upto ncards! @ cardlist!(i)\,\} \\
\t1		= \{\,j: 1 \upto hwm | dates(j) = today? @ names(j)\,\}
\end{schema}
The set on the right-hand side of the equation contains all the names
in the $names$ array for which the corresponding entry in the $dates$
array is $today?$.
The program code for $Remind$ uses a loop to examine the entries one by one:

%\pagebreak
\[
	\Procedure Remind(today: DATE; \\
\t5			\Var cardlist:\;\Array [1\upto {}]\Of NAME; \\
\t5			\Var ncards: INTEGER); \\
\t1		\Var j: INTEGER; \\
	\Begin \\
\t1		ncards := 0; j := 0; \\
\t1		\While j < hwm \Do\;\Begin \\
\t2			j := j + 1; \\
\t2			\If dates[j] = today \Then\;\Begin \\
\t3				ncards := ncards + 1; \\
\t3				cardlist[ncards] := names[j] \\
\t2			\End \\
\t1		\End \\
	\End;
\]
The initial state of the program has $hwm = 0$:
\begin{schema}{InitBirthdayBook1}
	BirthdayBook1
\where
	hwm = 0
\end{schema}
Nothing is said about the initial values of the arrays $names$ and
$dates$, because they do not matter. If the initial concrete state
satisfies this description, and it is related to the initial abstract
state by the abstraction schema $Abs$, then
\begin{argue}
	known \\
\t1	= \{\,i: 1 \upto hwm @ names(i)\,\} & from $Abs$ \\
\t1	= \{\,i: 1 \upto 0 @ names(i)\,\} & from $InitBirthdayBook1$ \\
\t1	= \emptyset. &			    since $1 \upto 0 = \emptyset$
\end{argue}
so the initial abstract state is as described by $InitBirthdayBook$.
This description of the initial concrete state can be used to write a
subroutine to initialize our program module:
\[
	\Procedure InitBirthdayBook; \\
	\Begin \\
\t1		hwm := 0 \\
	\End;
\]

In this direct refinement, we have taken the birthday book specification
and in a single step produced a program module which implements it.
The relationship between the state of the book as described in the
specification and the values of the program variables which represent
that state was documented with an abstraction schema, and this allowed
descriptions of the operations in terms of the program variables to be
be derived. These operations were simple enough to implement immediately,
but in a more complex example, rules of operation refinement could be
used to check the code against the concrete operation descriptions.

\paragraph{Exercise.}
Implement the class register you specified earlier. Use two arrays
\[
	names:\;\Array [1\upto{}] \Of NAME; \\
	finished:\;\Array [1\upto{}] \Of (Yes, No);
\]
Document:
\begin{enumerate}
\item	The concrete state space.
\item	The abstraction relation.
\item	The concrete operations.
\end{enumerate}

\section{A simple checkpointing scheme}

This example shows how refinement techniques can be used at
a high level in the design of systems, as well as in
detailed programming. 
What we shall call a {\em database\/} is simply a function from
addresses (modelled by the set $ADDR$) to pages of data ($PAGE$):
\begin{zed}
%%	[ADDR, PAGE] \\
	DATABASE == ADDR \fun PAGE.
\end{zed}
This definition written with the sign $==$ introduces $DATABASE$ as an
abbreviation for the set of functions from $ADDR$ to $PAGE$.  We shall
be looking at a system which -- from the user's point of view --
contains two versions of a database
\begin{schema}{CheckSys}
	working, backup: DATABASE
\end{schema}
This schema has no predicate part: it specifies that the two observations
$working$ and $backup$ may be any databases at all, and need not be related.

Most operations affect only the working database. For example,
it is possible to access the page at a specified address:
\begin{schema}{Access}
	\Xi CheckSys \\
	a?: ADDR \\
	p!: PAGE
\where
	p!= working(a?)
\end{schema}
This operation takes an address $a?$ as input, and produces as its output
$p!$ the page stored in the working database at that address.  Neither
version of the database changes in the operation.

It is also possible to update the working database with a new page:
\begin{schema}{Update}
	\Delta CheckSys \\
	a?: ADDR \\
	p?: PAGE
\where
	working' = working \oplus \{a? \mapsto p? \} \\
	backup' = backup
\end{schema}
In this operation, both an address $a?$ and a page $p?$ are supplied as
input, and the working database is updated so that the page $p?$ is now
stored at address $a?$.  The old contents of the address are lost.

There are two operations involving the back-up
database. We can take a copy of the working database: this
is the $CheckPoint$ operation: 
\begin{schema}{CheckPoint}
	\Delta CheckSys
\where
	working' = working \\
	backup' = working
\end{schema}
We can also restore the working database to the state it had
at the last checkpoint:
\begin{schema}{Restart}
	\Delta CheckSys
\where 
	working'=backup \\
	backup' = backup
\end{schema}
This completes the specification of our system, and we can begin to
think of how we might implement it.
A first idea might be really to keep two copies of the database, so
implementing the specification directly.
But experience tells us that copying the entire database is an expensive
operation, and that if checkpoints are taken frequently, then the computer
will spend much more time copying than it does accessing and updating
the working database.

The mathematics cannot make observations like this one
automatically for us, but by allowing the
specification to be expressed precisely and abstractly,
mathematical techniques can help the designer to carry out
this kind of analysis, perhaps by calling to mind other, similar systems.

A better idea for an implementation might be to keep only one complete
copy of the database, together with a record of
the changes made since creation of this master copy.
The master copy consists of a single database:
\begin{schema}{Master}
	master: DATABASE
\end{schema}
The record of changes made since the last checkpoint is a
{\em partial function} from addresses to pages: it is partial because
we expect that not every page will have been updated since the last
checkpoint.
\begin{schema}{Changes}
	changes: ADDR \pfun PAGE
\end{schema}
The concrete state space is described by putting these two parts together:
\begin{schema}{CheckSys1}
	Master \\
	Changes
\end{schema}
How does this concrete state space mirror our original abstract view?
The master database is what we described as the back-up, and
the working database is $master \oplus changes$, the result
of updating the master copy with the recorded changes. We can record
this relationship with an abstraction schema:
\begin{schema}{AbsDB}
	CheckSys \\
	CheckSys1
\where  
	backup = master \\
	working = master \oplus changes
\end{schema}
The notation $master \oplus changes$ denotes a function which agrees
with $master$ everywhere except in the domain of $changes$, where it
agrees with $changes$.

How can we implement the four operations? Accessing a page
at address $a?$ should return
$working(a?) = (master \oplus changes)(a?)$,
so a valid specification of $Access1$ is as follows:
\begin{schema}{Access1}
	\Xi CheckSys1 \\
	a?: ADDR \\
	p!: PAGE
\where
	p! = (master \oplus changes)(a?)
\end{schema}
But we can do a little better than this:
if $a? \in \dom changes$, then
\[
	(master \oplus changes)(a?) = changes(a?),
\]
and if $a? \notin \dom changes$, then
\[
	(master \oplus changes)(a?) = master(a?).
\]
So we can use operation refinement to develop the operation further; it is
implemented by
\[
	\Procedure Access(a: ADDR; \Var p: PAGE); \\
\t1		\Var r: RESULT; \\
	\Begin \\
\t1		GetChange(a, p, r); \\
\t1		\If r \ne found \Then \\
\t2			ReadMaster(a, p) \\
	\End;
\]
What are the operations $GetChange$ and $ReadMaster$? We need give only
their specifications here, and can leave their implementation to a
later stage in the development.
$GetChange$ operates only on the $changes$ part of the state;
it checks whether a given page is present,
returning a report and, if possible, the page itself:
%% \begin{zed} RESULT ::= found | not\_present \end{zed}
\begin{schema}{GetChange}
	\Xi Changes \\
	a?: ADDR \\
	p!: PAGE \\
	r!: RESULT
\where
	(a? \in \dom changes \land \\
\t1		p! = changes(a?) \land \\
\t1		r! = found) \lor \\
	(a? \notin \dom changes \land \\
\t1		r! = not\_present)
\end{schema}
As you will see, this is a specification which could be structured nicely
with the schema $\lor$ operator.
The $ReadMaster$ operation simply returns a page from the $master$ database:
\begin{schema}{ReadMaster}
	\Xi Master \\
	a?: ADDR \\
	p!: PAGE
\where
	p! = master(a?)
\end{schema}
For the $Update$ operation, we want $backup' = backup$, so
\[
	master' = backup' = backup = master.
\]
Also
$working' = working \oplus \{a? \mapsto p? \}$,
so we want
\[
	master' \oplus changes'
		= (master \oplus changes) \oplus \{a? \mapsto p?\}.
\]
Luckily, the overriding operator $\oplus$ is associative: it
satisfies the law
\[
	(a \oplus b) \oplus c = a \oplus (b \oplus c).
\]
If we let $changes' = changes \oplus \{a? \mapsto p?\}$, then
\begin{argue}
	working' \\
\t1	= working \oplus \{a? \mapsto p?\} &	spec.\ of $Update$ \\
\t1	= (master \oplus changes) \oplus \{a? \mapsto p?\} & 
						from $AbsDB$ \\
\t1	= master \oplus (changes \oplus \{a? \mapsto p? \}) & 
						associativity of $\oplus$ \\
\t1	= master' \oplus changes'. & 		spec.\ of $Update1$
\end{argue}
and the abstraction relation is maintained.
So the specification for $Update1$ is
\begin{schema}{Update1}
	\Delta CheckSys1 \\
	a?: ADDR	 \\
	p?: PAGE
\where
	master' = master \\
	changes' = changes \oplus \{a? \mapsto p?\}
\end{schema}
This is implemented by an operation $MakeChange$ which has the same effect
as described here, but operates only on the $Changes$ part of the state.

For the $CheckPoint$ operation, we want $backup' = working$,
so we immediately see that
\[
	master' = backup' = working = master \oplus changes.
\]
We also want $working' = working$, so 
\[
	master' \oplus changes' = master \oplus changes = master'.
\]
This equation is solved by setting $changes' = \emptyset$, since
the empty function $\emptyset$ is a right identity for $\oplus$,
as expressed by the law
\[
	a \oplus \emptyset = a.
\]
So a specification for $Checkpoint1$ is
\begin{schema}{CheckPoint1}
	\Delta CheckSys1
\where
	master' = master \oplus changes \\
	changes' = \emptyset
\end{schema}
This can be refined to the code
\[
	MultiWrite(changes); ResetChanges
\]
where $MultiWrite$ carries out the updating of the $master$ database,
and $ResetChanges$ sets $changes$ to $\emptyset$.

Finally, for the operation $Restart1$, we have $backup' = backup$, so we need
$master' = master$, as for $Update$.  Again, we want
\[
	master' \oplus changes' = master',
\]
this time because $working' = backup$,
so we choose $changes' = \emptyset$ as before:
\begin{schema}{Restart1}
	\Delta CheckSys1
\where
	master' = master \\
	changes' = \emptyset
\end{schema}
This can be refined to a simple call to $ResetChanges$.

Now we have found implementations for all the operations
of our original specification.
In these implementations, we have used two new sets of operations,
which we have specified with schemas but not yet implemented.
One set, $ReadMaster$ and $MultiWrite$ operates on the $master$ part of
the concrete state, and the other, containing $MakeChange$, $GetChange$,
and $ResetChanges$, operates only on the $changes$ part of the state.
The result is two new specifications for what are in effect modules
of the system, and in later stages they can be developed independently.
Perhaps the $master$ function would be represented by an array of
pages stored on a disk, and $changes$ by a hash table held in
main store.

The mathematical method can describe data structures
with equal ease, whether they are held in primary or
secondary storage. It describes operations in terms of their
function, and is indifferent to whether the execution takes
microseconds or hours to finish. Of course, the designer
must be very closely concerned with the capabilities of the
equipment to be used, and it is vital to distinguish primary
storage, which though fast has limited capacity, from the
slower but larger secondary storage. But we regard it as a
strength and not a weakness of the mathematical method that
it does not reflect this distinction.  
By modelling only the functional characteristics of a software
module, a mathematical specification technique encourages a healthy
{\em separation of concerns}:
it helps the designer to focus his or her attention on functional aspects, and
to compare different designs, even if they differ widely in
performance.

\twinkle

There has been space in this article for only a sketch of the
refinement process which leads from a Z specification to a working
program.  There has been no space to state explicitly the facts which
must be proved to show that a data refinement is valid, and we have
only touched on the idea of operation refinement, in which a
specification written in the notation of predicate logic is turned
into a program written in a programming language.  But I hope that an
important message has come across.

This message, which might be taken as the creed of a mathematical
approach to programming, is that the development of a program and the
proof of its correctness are not tasks which should be undertaken
separately. It is notoriously difficult to give even an informal proof
of correctness for a program which has been developed without such a
proof in mind, so rigorously-checked programs and their proofs must be
developed together.  More importantly, however, the constraints on the
development of the program imposed by the need to prove its
correctness can help to guide the process of development.  In the
checkpoint example, we saw how the rules of refinement allowed us to
find the descriptions of the concrete operations almost by
calculation.  This is the real power of mathematical methods: their
ability to systematize our understanding of programming and make us
more articulate in explaining specifications and programs to others.
This is the reason why mathematical methods are becoming more
important in every-day programming, even where risks to life and
property do not make a formal, mathematical proof of correctness into
a moral necessity.

\section*{Acknowledgements}

The author is grateful to M. A.~McMorran, S.~Powell and J.
B.~Wordsworth of IBM United Kingdom Laboratories, and to T.~Clement of
the University of Manchester for helpful comments on the paper, and to
Oriel College, Oxford and Rank Xerox (UK) Ltd.\ for financial
support.

\begin{thebibliography}{99}
\raggedright

\bibitem{Date}
DATE, C.~J.:
`An Introduction to Database Systems', third edition
(Addison-Wesley, 1981).

\bibitem{CAVIAR}
FLINN, L.~W., and S{\O}RENSEN, I.~H.:
`CAVIAR: A Case Study in Specification'.
In `Specification Case Studies', ed. I.~J. Hayes
(Prentice-Hall International, 1987).

\bibitem{Hayes}
HAYES, I.~J.:
`Applying Formal Specification to Software Development in Industry'.
In `Specification Case Studies', ed. I.~J. Hayes
(Prentice-Hall International, 1987).

\bibitem{UNIX}
MORGAN, C.~C., and SUFRIN, B.~A.:
`Specification of the UNIX Filing System'.
In `Specification Case Studies', ed. I.~J. Hayes
(Prentice-Hall International, 1987).

\bibitem{Jones80}
JONES, C. B.:
`Software Development: A Rigorous Approach'
(Prentice-Hall International, 1980).

\bibitem{Jones86}
JONES, C.~B.:
`Systematic Software Development using VDM'
(Prentice-Hall International, 1986).

\bibitem{Spivey}
SPIVEY, J.~M.:
`Understanding Z: A Specification Language and its Formal Semantics'
(Cambridge University Press, 1987).

\end{thebibliography}

\end{document}
